Acknowledgments
Special thanks to The John S. and James L. Knight Foundation, The Tow Foundation,
and the team from the Tow Center for Digital Journalism at the Columbia University
Graduate School of Journalism. Our gratitude to Andy Carvin and Charles Sennott for
valuable guidance, and to Abigail Ronck for copy editing this Tow/Knight brief. Without
all of your support this project would not have been possible.
\chapter{Introduction}
By Ann Cooper and Taylor Owen
Throughout the twentieth century the core proposition of foreign correspondence was to
bear witness—to go places where the audience couldn’t and report back on what
occurred. Foreign correspondents have long been our interpreters of global events.
Three interrelated trends now challenge this position. First, citizens living through events
can tell the world about them directly via a range of digital technologies, without the need
for the journalist intermediaries who were essential in the past to report and distribute the
news. Second, journalists have the ability to report on some events, particularly breaking
news, without physically being there—by immersing themselves in streams of content,
whether live video feeds from cell phones, Twitter feeds, or blog posts. Finally, the
financial pressures that digital technology have brought to legacy news media have
forced many to close their international bureaus. Scores of the traditional foreign
correspondents who worked in those bureaus have moved on to other careers, or to new
jobs at digitally native media.
In this age of post-legacy media, local reporters, activists, ordinary citizens—and
traditional foreign correspondents—are all now using digital technologies to inform the
world of breaking news, and to offer analysis and opinions on global trends. In 2012,
calling the nontraditional forms of coverage they identified The New Global Journalism.^{\href{#endnotes}{1}}
The new practices require new skills. The population of experts who report and comment
on the news has expanded to include eyewitnesses who can film, or photograph, or write
about events and post their work immediately online. Verifying the authenticity of these
digital sources has become an important part of the skillset for international reporting.
Also newly important is the ability to do immersion social-media reporting. And in this
new age of government surveillance, traditional correspondents and new digital reporters
must train themselves to use security tools that will keep their work and their sources
secure.
In late 2013 the Tow Center gathered several practitioners and students of the new global
journalism for a discussion of this changing world. A theme quickly emerged: that ``being
there,'' meaning the hallowed belief that foreign correspondence had to come directly
from the eyewitness reporter on the ground, was no longer the only reliable source for
reporting from abroad.
In fact, there have always been stories that foreign correspondents have had to find a way
to report on without actually ``being there.'' China’s government, for years, refused to
accredit foreign correspondents, so Western news organizations set up listening posts in
Hong Kong. Still today, North Korea’s highly restrictive regime forces news
organizations to cover the country from South Korea. Similarly, government restrictions
or the high level of physical danger have forced correspondents at times to cover conflicts
in Somalia, Nagorno-Karabakh, Syria, and other hot spots from neighboring outposts.
These solutions are limiting and unsatisfying, but in this digital age it’s far more possible
to do reliable reporting and good analysis without always being on location. A hybridreporting
project like the site Tehran Bureau, based in London and described by its
founder in this report, can add depth to our understanding—particularly when reporting
on authoritarian regimes.
The digital journalist uses a host of new electronic sources, tools, and practices that are
now part of the global reporting landscape. Digital journalists would argue that in the
right circumstances, these tools enable them to offer as clear and informed a report as
what the journalist on the ground can produce—sometimes even clearer, because they
may have access to a broader spectrum of material than a field reporter.
In mainstream newsrooms, though, there is still significant skepticism about digital’s
impact on foreign reporting. Many see it as the end of the era when a reporter could
spend a full day—or days or weeks—reporting in the field before sitting down to write.
That traditional foreign correspondent model served audiences well, bringing them vivid
accounts of breaking news and nuanced analysis of longer-term developments.
At the Tow Center, we believe that both forms of reporting are vital, that both are
necessary to help all of us understand the world. A goal of this report is to narrow or
eliminate the divide between the two, and in this spirit we lay out several objectives.
First, our authors work to provide a clear picture of this new reporting landscape: Who
are the primary actors, and what does the ecosystem of journalists, citizens, sources,
tools, practices, and challenges look like? Second, we urge managers at both mainstream
and digital native media outlets to embrace both kinds of reporting, melding them into a
new international journalism that produces stories with greater insight. Third, we hope to
show the strengths in traditional and digital foreign reporting techniques, with a goal of
defining a hybrid foreign correspondent model—not a correspondent who can do
everything, but one open to using all reporting tools and a wide range of sources. Finally,
we outline governance issues in this new space—legal and operational—with an aim to
help journalists report securely and independently in this digital age.
We approach these issues through five chapters, whose authors include journalists from
both digital-native and mainstream media, as well as a communications scholar and a
media producer for a human rights organization. While each writes from a different
vantage, the overlapping insights and conclusions begin to redefine both the edges and
heart of international reportage now.
The Virtual Eyewitness
While some doors in legacy media have closed, others have been opened by digital
innovators such as Kelly Golnoush Niknejad, who founded Tehran Bureau from her
parents’ living room in Massachusetts. Starting from a network of contacts she had
developed on Yahoo! Messenger and Gchat, Niknejad has developed a vibrant news site
on an authoritarian regime, with contributions from a pool of citizen journalists inside the
country, as well as from those representing the Iranian diaspora.
Niknejad argues that her model can be more effective than ``being there'' in an
authoritarian regime such as Iran, where foreign correspondents are closely monitored
and end up censoring themselves to avoid expulsion. Her model is a vivid example of
how new and old media can work together to report in challenging areas.
The Foreign Desk in Transition
There are still legacy foreign desks with traditional foreign correspondents at some major
U.S. media. But the legacy newsrooms are also adapting to the digital era. Anup Kaphle,
digital foreign editor at the Washington Post, describes experiments like WorldViews, his
paper’s international blog, which relies on dispatches from correspondents abroad, as
well as pieces by two in-house journalists who never leave Washington to report.
Kaphle also looks at digitally native sites like BuzzFeed, which are opening bureaus in
foreign capitals. Their commitment to international news is brand new, but like the
legacy foreign desks, they are also grappling with how to shape the foreign desk of the
future.
Eight Tactics for the Digital Foreign Correspondent
During the 2011 Arab uprisings in Tunisia, Libya, Egypt, and Syria, the way we learned
about and reported on international events was upended. Alongside the CNN cameras
shooting from roofs of hotels, citizens were tweeting, posting video to YouTube, and
organizing on Facebook. All this information presented enormous challenges: how to
make sense of it, and whose reporting could you trust?
To Andy Carvin, at the time a senior product manager for online communities at NPR,
these twin problems of context and trust sounded a lot like an opportunity for good
journalism. Using traditional values and practices, he set out to monitor the digital
information flow and separate fact from rumor and speculation. In doing so, he and his
team—including Saudi journalist Ahmed Al Omran—developed practices for verifying
sources, making corrections, and dealing with anonymity on social media.
Al Omran, now a correspondent in the Wall Street Journal’s Saudi Arabia bureau,
outlines eight tactics for journalists who want to mine social media and other digital
sources for international reporting.
David Versus Goliath
One of the core attributes of the digital media ecosystem is that it is decentralized. Where
traditional foreign desks operate as hierarchical organizations with management, editors,
and levels of journalists, the digital media world can sometimes look like the Wild West.
This can leave citizen journalists more vulnerable to repressive tactics, including the
increased use of digital tools by governments seeking to stifle speech.
One group working through these questions is 140journos, a network of Turkish citizen
journalists who are seeking to counterbalance the country’s censored national media.
Columbia University communications Ph.D. candidate Burcu Baykurt explores what
140journos and others are doing in the new digital space, and what threats they—and
more traditional correspondents— face from governments.
A Professional Kinship: Journalism and Advocacy
With media tools like video recording and Internet transmission now widely available,
people and institutions all over the globe have the ability to commit journalistic acts.
Advocacy organizations such as Human Rights Watch and WITNESS have developed
digital skills put to practice with the aim of informing the public—but also aggressively
advocating for change.
Jessie Graham, formerly a public radio journalist and now senior media producer at
Human Rights Watch, explores the shifting line between journalism and advocacy
organizations. Advocates once depended on media to report on their research; now they
can reach the public directly. Human Rights Watch and others also hire journalists,
particularly photographers, to help with their work. The journalists’ reporting may end up
on an advocacy website—and in the columns of mainstream media.
Together, the authors of this volume offer a glimpse into the new global, digital
journalism. Call them innovators, social media experts, or activists—above all, they are
excellent journalists who are redefining reporting in the digital age.
\chapter{Being There: The Virtual Eyewitness}
By Kelly Golnoush Niknejad
At the turn of this century, I started noticing a torrent of messages from Iranian strangers
each time I logged into my Yahoo! Messenger account. I was a courts reporter in San
Diego and thought the dozens of daily chat requests were coming from up north in Los
Angeles, where the largest concentration of Iranians live outside their homeland. But an
adjective in one message struck me. ``Take a look at my photo,'' one guy implored, after
several unsuccessful attempts at getting my attention. ``I’m really quite ghashangh.''
Ghashangh? The word, meaning pretty in Farsi, was something my grandmother in
Tehran might use to describe a young boy. What macho Iranian-American would refer to
himself that way?
I wrote back. ``Are you in Tehran?'' I asked on a hunch.
``Yes,'' he said, ``aren’t you?''
As it turned out, just about all the messages were streaming in from Iran—and not just
Tehran. All were male, most of them very young, in their late teens or early twenties,
looking for a date. They didn’t believe I was in the United States and well outside their
demographics. Those circumstances made it difficult to have a proper conversation, or
ask the kind of questions I wanted as a reporter, but it gave me a fascinating look inside
the country.
That digital opening offered me a view into Iran that was more enlightening than any
journalism coming out of the country at the time. Five years later, by the time I left for
had started migrating from Yahoo! to Gmail. Now I noticed messages coming from the
elite—university students and faculty, even an increasing number from within the cadre
of the ruling establishment.
In fact, checking emails one day, I thought it peculiar that two of the bright green circles
that lit up on my Gmail instant chat belonged to officials: a relatively high-ranking
American and an Iranian one. Though universes apart in distance and ideology, they were
next to each other, live in my email account, and easily within my grasp. I began to think
of how I, as a journalist, could connect those virtual dots. The result was Tehran Bureau,
the blog I launched from my parents’ living room in Massachusetts in 2008.
The blog’s name found its genesis in a conversation with David Remnick of The New
Yorker. Remnick attributed a dearth of in-depth reporting on Iran to a lack of any news
organization with ``a real bureau there.'' Tehran Bureau was a response to shrinking
foreign news coverage in the United States. It defied tradition. I was not actually in Iran,
but thanks to digital reporting tools the blog could tackle media stereotypes and fill in
some of the gaps left by newspapers’ scaling back overseas staffs.
As it turned out, the virtual bureau became the new frontline. In a very short time, Tehran
Bureau went from occupying a small perch on blogspot.com to a partnership with PBS
Frontline. It now reaches a larger audience at The Guardian, where Tehran Bureau is
hosted on the newspaper’s popular website and featured occasionally in print. At its
Guardian home, a typical Tehran Bureau post gets anywhere from 10,^{\href{#endnotes}{000}} to 50,^{\href{#endnotes}{000}} hits,
though some stories—like a recent piece about modeling—drew ten times as much
traffic.
During this growth process, Tehran Bureau went through another, subtler, but significant
change. We went from being thousands of miles away from the story to being there—
literally. In time, we transitioned from doing all of our reporting from a U.S. post to
growing a huge pool of reporters, editors, and fact-checkers inside the country.
It happened organically, in part because ``the old frame for thinking of ‘natives’ who are
staying back home and ‘natives’ who have left home doesn’t quite work anymore,'' said
Iraj Omidvar, an Iranian-American professor who teaches English, technical
communication, and media in Atlanta. Thanks to digital technology, ``Iranians in Iran and
abroad are tied in unprecedented ways that are dramatically changing a wide range of
cultural phenomena.'' Not the least of which is journalism.
``The Tremor in the Air''
A widely held belief about new media and traditional foreign correspondence suggests
that the two are mutually exclusive. On the one hand, a foreign correspondent steeped in
the best practices of our profession heads off to cover a war, revolution, or natural
disaster in some far-flung place, notebook in hand, ready to bear witness. On the other, a
pajama-clad blogger pontificates from an armchair in Maryland about events thousands
of miles away. It’s all at his fingertips: the flood of news, images, and video captured by
citizen journalists on the ground. Why would you need a correspondent to be there as an
eyewitness too?
Well, because, ``No search engine gives you the smell of a crime, the tremor in the air, the
eyes that smoulder, or the cadence of a scream,'' Roger Cohen argued in a June 2009
column for the New York Times, making an eloquent appeal for traditional foreign
correspondence.^{\href{#endnotes}{2}} Cohen had just been ousted from Iran, from where he had continued to
report on the violent aftermath of the presidential election even after he was stripped of
his press card.
``No news aggregator tells of the ravaged city exhaling in the dusk, nor summons the
defiant cries that rise into the night,'' Cohen wrote. ``No miracle of technology renders the
lip-drying taste of fear. No algorithm captures the hush of dignity, nor evokes the
adrenalin rush of courage coalescing, nor traces the fresh raw line of a welt.''
No doubt Cohen argued a strong point about being there on the scene, recording
everything firsthand. But now, using digital technology, there is another way to grasp the
granular and authentic feel of the streets. This process uses new online tools, not to
circumvent the most sacred principles of journalism, but to advance them—especially
when reporting on authoritarian countries like Iran, Ethiopia, North Korea, or the United
Arab Emirates, to name just a few. New media allows journalists to cast their nets wider
than ever in some of the most underreported places in the world.
At Tehran Bureau we gather information from ordinary people, charting the trends in
society from the ground up. By remaining anonymous and going under the radar, we can
penetrate a closed society whose members have largely withdrawn into tight-knit units.
We operate without official access, beyond the controls and spin the government uses to
manipulate or influence journalists in traditional Tehran bureaus. Thus, new media allows
us to do the kind of independent reporting that is virtually impossible for a physical news
bureau inside Iran.
The Price of Access
In my first three years as a staff writer for a news service in California and a small
newspaper in Massachusetts, I was on the scene of just about every story I covered. It’s
what made me fall in love with reporting. Once, in those early days, I heard two national
reporters at the Los Angeles Times commiserating about a new editor. He’d ordered them
to report using the phone in all but the most exceptional cases. Sacrilegious, I thought.
Even then.
But Los Angeles is not Tehran, or Damascus, or Hanoi, or any of a number of other
capitals where governments keep tight reins on foreign correspondents—not to mention
their own journalists. The press freedoms enjoyed in American and European societies
yield a high level of reliability in the coverage of news in the Western world. Not so
when it comes to partially closed societies such as Iran, where foreign correspondents can
face a series of obstacles that prevent unfettered newsgathering.
The first hurdle for foreign journalists seeking to cover authoritarian regimes is often the
necessity of getting official accreditation from the government. In Iran this can involve
long and difficult negotiations with government officials. Those who make it through the
accreditation process are then subject to constant monitoring by the government. This is
done by requiring visiting reporters to employ minders from the Ministry of Culture and
Islamic Guidance. Interviews must be arranged through the minder, who usually sits next
to the reporter during these sessions and acts as a translator.
Foreign correspondents often fail to tell their viewers and readers about the omnipresent
minders, though some eventually reveal details after they leave their reporting posts.
Azadeh Moaveni, an Iranian-American who reported for Time magazine from Iran for
several years, later admitted to being constantly ``trailed by a hook-nosed security agent,
bullied to inform on my sources, and threatened with prosecution for ‘endangering
national security.’ '' In her book, Honeymoon in Tehran, she describes Mr. X, her secret
government minder, as ``perhaps the most important person in my Iranian life.''^{\href{#endnotes}{3}}
There’s another layer of control too. Minders have shadow minders, a long-standing
practice in Iranian statecraft, where somebody is watching the person watching you—just
to make sure that your personal minder does the government’s bidding.
For the few foreign journalists with access to the country, the pressures don’t end with
getting accreditation and having to hire a minder who’s on a government payroll. There is
always the danger of losing the hard-won government credentials, and to avoid that, it’s
necessary to self-censor your journalism. This is true for all foreign correspondents, but
particularly for Iranians or those with Iranian spouses who are filing from inside the
country for foreign news outlets. They work under even greater duress because their
families are virtual hostages.
The latest example involved Washington Post Tehran correspondent Jason Rezaian, who
has dual U.S.-Iranian citizenship, and his journalist wife Yeganeh Salehi, an Iranian
citizen. They, along with a dual-nationality photographer, were arrested in July of 2014.
The photographer, who was not identified publicly at the family’s request, was released a
month later, while Rezaian and Salehi remained incarcerated but uncharged. The Post
described the detentions as evidence of yet another high-level internal political struggle—
though in reality they follow a long-standing pattern of dealing with foreign
correspondents.
When Nazila Fathi, a local correspondent for the New York Times, fled Iran in 2009, she
explained how the Iranian establishment was more restrictive with respect to foreign
reporting than the domestic press. Certain topics, such as executions, she said, were so
sensitive that she was prohibited from writing about them, even though some of their
accounts were published in Iranian newspapers. Fathi also learned that Iran’s intelligence
services draw no line between work and private life. In her case, the government’s
monitoring extended beyond listening to her calls and reading her emails. As she recounts
in a forthcoming memoir, the woman who looked after her toddlers in her home was a
government spy.
The experience of Guardian correspondent Dan De Luce is illustrative of the cost of
bucking the system in countries like Iran. De Luce arrived in Tehran in January of 2003.
Like Fathi and Moaveni, he was working there when conditions were actually pretty
good. Mohammad Khatami, a reformist, was president. It was a time of relative openness
for the press. And yet, even under Khatami it was difficult for a foreign correspondent to
cover the country.
``Difficult in a subtle way,'' De Luce explained to me back then. On the surface there
appeared to be a certain openness, but in effect ``foreign journalists are on a tight leash by
the visa regime,'' he said. ``It’s very difficult for a reporter to get into the country, and
once in, there is the sensitive matter of getting one’s visa renewed every three months.
That is a check that discourages journalists from pursuing certain stories.''
The government also discourages controversial stories by intimidating the translators and
fixers, said De Luce. They’re very closely watched, their telephones are tapped, and they
are interrogated by the government on a regular basis. ``I knew that whatever I did was
going to be an open book.''
Translators have a way of not pursuing a correspondent’s story without coming out and
saying ``no'' directly, he explained. Sometimes, even when translators are willing to cross
the line, the correspondent may not be willing to take that risk on behalf of the fixer or
translator.
Language is another barrier to good reporting. Foreign journalists in Iran or other
restrictive countries often do not speak the local language, ``so they miss a lot,'' said De
Luce. ``They speak to each other, to other foreigners, and diplomats.'' And yet native
language skills can also be a hazard. The government is particularly paranoid about
Iranian expats, who can pass through the country easily with their Iranian passports and
fluent language abilities.
``A lot of what is going on there never sees the light of day,'' De Luce said. ``What goes
on is as mysterious as the goings-on in the Kremlin during the Soviet era.'' Add to that
the paranoia of the Iranian government, which tends to view foreigners as spies or
fomenters of revolution, he offered. And when it comes to its own citizens, the
government believes it can operate with impunity. De Luce cited the Iranian-Canadian
photojournalist Zahra Kazemi, who was bludgeoned to death in the summer of 2003
while in police custody. That would have never happened to him or another foreigner, he
said.
What did happen to De Luce is more typical of the fate of the foreign correspondent who
overstepped limits in the former Soviet Union or China. Following the 2003 earthquake
in Bam, a city in the Iranian province of Kerman, most correspondents waited in Tehran
for earthquake news from formal channels. De Luce and his wife, who wrote for an Irish
newspaper, signed up to go to the earthquake region as volunteer rescue workers. Once
there, they documented the frustrations and devastation of the earthquake survivors—
until the government expelled them. Their crime was being too aggressive in getting the
story.
Not one foreign journalist based in Tehran was willing to protest his expulsion, said De
Luce; all were too concerned with maintaining their own access. He understands, though.
Is it such a success to be ``really gung ho'' and get kicked out and not report anything at
all, he asks, or, ``Isn’t it better to have someone there?'' De Luce’s response: There is no
right answer.
Going Vertical
Playing by the rules may lead to flawed reporting, but it gets at ``one version of the truth,''
said a former foreign editor at a major U.S. newspaper. But that’s a shaky argument.
When the self-censored reporting of a Tehran-based foreign correspondent is published
by a mainstream newspaper of record, and cited widely, it becomes much more than just
one competing strain of a story. In fact, until the recent proliferation of blogs, that one
self-censored version was often the only one that filled the news vacuum.
I’m not advocating that news organizations abandon foreign bureaus in authoritarian
countries—only that they supplement their reporting from those places and use their
websites as platforms to present deeper work and multiple voices that don’t all fit in the
daily print paper.
Bill Rempel, a former senior editor and investigative reporter at the Los Angeles Times,
once told me that ideally he’d like big news developments from Iran reported
independently, by separate reporters, even if they turned up different stories or
conclusions. Rempel said the stories could play side by side, so that readers could take a
look and make up their own minds.
Using new digital tools and resources makes that possible and affordable. And doing it
means we no longer have to accept self-censored, misleading reporting, like the kind that
helped Khatami—the president when Dan De Luce was thrown out of Iran—receive such
glowing coverage from foreign correspondents who played by the rules.
Reading the news during the Khatami era, I felt there was a gap between what I saw in
mainstream media here and public opinion in Iran. I took up this theme in my master’s
project at Columbia Journalism School. The problems I encountered helped explain why
I thought I was getting a clearer and more nuanced picture reporting on Iran from New
York.
My not being there had distinct advantages in the Internet age, when technology opened
up many new avenues of communication and allowed new voices to be heard. In an
article for Nieman Reports a few months after Tehran Bureau was launched, I explained
that one of my primary motivations in setting up ``the virtual Iran beat'' was to assemble a
staff of reporters and editors who spoke Farsi.^{\href{#endnotes}{4}} This meant we could tap into a more
extensive network and speak to more Iranians, even if we were not based in Tehran. And
free of the filters that limit Internet access from within Iran, we could read Iranian
bloggers—those who write in Iran and those who live in exile.
That was the idea, anyway. Before the 2009 post-election crackdown, there were
hundreds, perhaps thousands of people, blogging about Iran. Most were very opinionated,
but even blogs with a strong point of view could be useful for possible story leads or a
different perspective on an issue. When Iran’s internal factional war spilled into the open,
even more valuable information began to appear online—often posted by one faction
seeking to discredit another. Tehran Bureau was well positioned to scan this wide range
of views, along with the Iranian press, to help inform the reporting by our staff.
To date, the online resources for us are relatively rich because Iranians are as much
plugged in online as any developed society. As the academic Omidvar explained,
``Networked digital media is permitting conversations that could never have taken place
before, between people who would have never come into contact with each other, with
often dramatic results that no one could have possibly predicted adequately.''
Toolkit Nuts and Bolts
Before launching Tehran Bureau, I set out to meet as many Iranians as I could. Since I
emigrated from Iran in 1984, I had lived, worked, or spent a lot of time in cities with
large Iranian communities like San Diego, Los Angeles, San Francisco, Berkeley,
London, and Dubai. I knew that ties between south Iran and what is now the United Arab
Emirates stretch back hundreds of years, with waves of migration that started long before
the 1979 revolution. Today there is a crosscurrent of Iranians heading to the Emirates,
and though they are largely middle class they offer a greater mix of opinions than you
might find in your social circles in Tehran. Most of them retain close ties to the
motherland. At Tehran Bureau, most of us are part of that kind of virtual community,
giving us a rich network to mine—as we did, thanks to new media, during the heavily
contested 2009 election and its bloody aftermath.
I launched the blog in November of 2008. The choice to use a blog format was a budget
issue; I had no money to create a more complex website, but from its beginning Tehran
Bureau was designed to publish reported stories, not thought pieces or opposition rants.
Our first dispatch from Tehran was a reaction to President Obama’s election victory. It
was cited by ABC News and the BBC World Service. Tehran Bureau went into
syndication soon thereafter. The first news organization to buy one of our stories was the
New York Times—all before the Iranian presidential election in June of 2009.
In February of that year, back in Boston, my sister brought up a name I hadn’t heard in
twenty-five years. ``Do you remember her?'' she asked. ``She was a classmate. She found
me on Facebook. See if you can find her.'' When I found this former classmate on
Facebook, I came across other mutual friends, many of them long-lost classmates from
the time of the Iran-Iraq war. I’d already been on Facebook for about three years and had
found a few profiles that appeared to have been posted from Tehran.
Four months before the election, these new profiles from my classmates turned out to be
part of a much larger wave—so large that it felt as if the whole of the Islamic Republic
had joined Facebook overnight. I followed the presidential campaign in part via status
updates on Facebook. It was like having a front row seat. One contact was working for
opposition candidate Mir Hossein Mousavi and had an insider’s view of much of what
was happening. The first signs of trouble came when that contact reported an attack on
the opposition candidate’s headquarters on the eve of the vote. And when YouTube
videos of demonstrations began spreading via Facebook, an Iranian neighbor was the first
to alert me.
I took to Twitter once our website was taken down by a powerful denial-of-service (DoS)
attack in June of 2009, presumably by an Iranian government proxy. The incumbent
regime of Mahmoud Ahmadinejad had declared victory (which many deemed fraudulent)
and now needed to cut off the election coverage in English; it did so swiftly by canceling
journalists’ visas or confining correspondents based there to their offices. Safely outside,
I could continue to get the story out. Even when the wrath of the regime spread through
cyberspace, news continued to trickle in via email, Skype, instant chat—even,
occasionally, the telephone. Text is relatively safe and easy to get out, even when the
Internet slows to a crawl.
Though limited to micro-blogging, I didn’t want to do away with gritty details or pare
harrowing accounts down to one-hundred-forty characters. On Twitter, I used full quotes
and punctuated as much as possible. I indicated when a quote came to an end, or when a
story would be carried by successive tweets. I reported these accounts from some of my
most trusted sources in the network I’d built. I avoided tweets from random strangers.
Actually, at that time, few Iranians were on Twitter, though the often-used term ``Twitter
Revolution'' did aptly capture the moment. Twitter was social networking stripped down
to its most fundamental. Reports came out in YouTube videos and firsthand accounts
from other channels making their way to that narrow intersection. And it was on Twitter
that some in the media already had a listening post.
Tehran Bureau’s Twitter reporting on the elections and the aftermath was cobbled into
narratives on the New York Times’ Lede blog and Andrew Sullivan’s The Dish. Our
Twitter feed @TehranBureau went from a few hundred followers to 19,^{\href{#endnotes}{000}} in two days.
There are more than 45,^{\href{#endnotes}{000}} now, though I rarely tweet anymore.
Tehran Bureau’s election coverage in 2009 is a typical example of how mainstream and
new media are coming together in journalism. Twitter and other social media have
become an integral part of getting the news out when a major event erupts somewhere in
the world. But what if that kind of synergy were systematic and employed more broadly,
beyond breaking news? What would it unearth? What could it mean for investigative
reporting in closed societies?
That’s still a largely untapped idea. Citizen journalism played an important role in Iran’s
2009 crisis, but when the story went underground, the citizens reporting it did too. They
generally lack the necessary perspective and investigative techniques to continue
chronicling events in a meaningful way. These skills remain crucial—perhaps even more
crucial—when the story is no longer on the street in the form of riots or demonstrations.
In Iran, the number of people able to report credibly from the inside diminished
significantly over the course of Mahmoud Ahmadinejad’s second term. Those with ties
outside the country left. Inside Iran, the ability of journalists to gather and disseminate
news was greatly hampered by the state’s ongoing crackdown on the press. But equally
significant was the lack of Iranian journalists trained in international standards of
reporting.
To continue and expand reporting from the ground, Tehran Bureau launched ``Iran
Standard Time.'' Adapted from the Washington Post’s ``Time Zones,'' it offers a view into
a doctor’s life, a taxi ride, and other aspects of everyday life inside the Islamic Republic.
We also expanded the commentary and analysis section, which may have allowed more
opinion to seep through, but it also helped give context to a complex story that wouldn’t
have been available otherwise. We have also devoted a large section to translating Farsilanguage
news sites. In the tumultuous post-election climate, the Iranian blogosphere was
often the best place to read between the lines and figure out what was going on; it’s
where we learned, for example, that hardline factions were going after each other in
public once their reformist targets were in jail or otherwise silenced.
The hardest and most rewarding part of the job is to discover and foster new talent,
especially at a distance. Traditional online training programs aimed at Iranian journalists
often don’t succeed in teaching how to report accurately and ethically. One problem is
that many of the journalists who undergo training are set in their ways and too proud to
take instruction. Another is that the training programs financed by Western governments,
including the United States, often just aren’t organized effectively.
At Tehran Bureau, I’m trying to get around some of those obstacles with a peer-to-peer
training program. This way we can calibrate the instruction to the level of the student. By
pairing students with seasoned practitioners, we try to produce professional content from
the start. Translators, who may be journalists in their own right, assist or take active part
in these working groups to bridge any language gaps and provide an extra layer of
reporting.
To keep everyone safe, we work anonymously—a policy that may be viewed as
anathema to good journalism. Iran operates on anonymity, though. And for our
correspondents, it’s essential for security. The openness and transparency that make for
good reporting practices in New York or Washington, D.C. are meaningless in Tehran—
even, I would argue, reckless.
As we expand the network, we recruit trusted reporters in different neighborhoods, and
eventually regions, with access to different strata of society. Even though our
correspondents don’t know each other, we can collaborate on stories through our shared
link outside the country. If a new reporter has a scoop, I can simultaneously assign the
same story to a second reporter with whom I have worked and trust. The two reports may
overlap and complement each other; if they don’t, we try to figure out why, a process that
may add more nuance to the reporting. Or, it may convince us that the story is flawed and
not useable.
Rather than framing journalism in the traditional newsgathering mold, which focuses on
the policy announcements of the ruling elite, Tehran Bureau covers Iran from the bottom
up. Our correspondents usually don’t have the press credentials required to attend
government press conferences and conduct interviews with high-level policymakers, but
they have unrivaled access to, and understanding of, the often unpredictable society in
which they live. This is not citizen journalism; this is professional journalism, done
undercover. They use notebooks and pens. They don’t carry cameras or other
conspicuous equipment. Emails cloaked with aliases provide additional cover.
More Left Undone
Still, we’ve only scratched the surface, hindered not by the government of Iran but by
lack of funding. The biggest obstacle to our reporting has been, and remains, money.
We’re not a think tank and don’t fill a policy prescription. Because we accept no money
from any government, religious faction, or interest group, it effectively cuts us off from
some of the richest sources of funding, including the U.S. government. Although we
work hard to stay above the political or ideological fray, most big foundations are
reluctant to support us because of the contentious subject matter. And as a board member
at one of these prominent organizations in New York put it to me, ``You’ll never get
funding because you’re Iranian.''
I have been fortunate to eke out a salary, first from PBS Frontline and now from The
Guardian, where we became part of the paper’s website in early 2013. ``While serious
independent journalism remains nearly impossibly in Iran,'' The Guardian said in
announcing our arrival, ``[Tehran Bureau] is able to provide original reporting throughout
its extensive list of contacts both inside and outside the republic, and to bring the voices
of ordinary people to an international audience.''^{\href{#endnotes}{5}}
I make our small budget stretch as far as it will go to pay editors, writers, and, when
possible, translators—most of whom have generously donated their time to make it
possible to pay more reporters. We are still looking for long-term funding for what has
already proven to be a valuable journalistic enterprise; the journalists who work for us
need other jobs, too, to survive.
In the meantime, according to editor and correspondent Oliver August of The Economist,
Iran remains ``the most underreported country in the world.'' It doesn’t have to be that
way, however, said Omidvar, the Atlanta professor. ``There is a massive, untapped—but
tappable—pool of Iranian talent for collecting, distributing, and evaluating information
on,'' he said. And with the right combination of online technologies and journalistic skill,
it can be done in ways that ``apparatuses of repression would never be able to counter.''
\chapter{A Toolkit: Eight Tactics for the Digital Foreign Correspondent}
By Ahmed Al Omran
Since the earliest days of journalism, new technologies have periodically changed
reporting, accelerating how we transmit information. (Think the telegraph, telex,
telephone, and satellite transmission.) Digital technologies have changed the speed
equation yet again, bringing new benefits and challenges.
As we’ve witnessed in the last few years, speedier transmission pushes some journalists
to prioritize being first—over being accurate—with the aim of scoring an exclusive story.
``It’s more difficult to verify what’s true and what may be shockingly false,'' wrote CNN
in a 2009 article about false social media reports of celebrity deaths.^{\href{#endnotes}{6}} This was nearly
four years before CNN and other major U.S. news organizations rushed to break ``news''
in the Boston Marathon bombing, only to learn the sources they relied upon were wrong.^{\href{#endnotes}{7}}
Perhaps the most profound impact on reporting is the opportunity that digital
technologies give to people everywhere to commit ``random acts of journalism,'' as
former NPR senior strategist Andy Carvin put it. Anyone who witnesses a breaking news
event can, with the right digital tools and Internet access, report on it instantly to the
world. With all the firsthand information flooding onto YouTube, Twitter, and other
platforms, the question of where this leaves professional journalists is one we deal with
daily. Internationally, it gives correspondents more potential sources than ever before. It
also demands that they take greater responsibility for verifying their reporting, for being
transparent about their newsgathering techniques, and for correcting their mistakes.
What follows are eight tactics—all based on traditional reporting principles but adapted
to new technology—to help the foreign correspondent remain reliable, trustworthy, and
authoritative in the digital age.
Finding Sources
When I began working at NPR in the summer of 2011, we were several months into a
series of popular uprisings in the Middle East. Tunisian president Zine El Abidine Ben
Ali had fled his country in January, and Egyptian president Hosni Mubarak was ousted in
February. By the time I arrived, NPR’s senior strategist, Andy Carvin, had already made
a name for himself as a pioneer using social media to cover these uprisings. ``The man
who tweeted the revolution,'' The Guardian called him.^{\href{#endnotes}{8}} Columbia Journalism Review
described Carvin as ``a living, breathing real-time verification system.''^{\href{#endnotes}{9}}
What garnered him all the attention was that Carvin was not a traditional foreign
correspondent sending news directly from Tahrir Square, but a Twitter maven who did
his reporting from NPR’s D.C. headquarters. Carvin didn’t replace NPR’s on-the-ground
correspondents, of course, but his work complemented theirs. And he showed that it was
possible, perhaps for the first time, to do serious reporting on a revolution without
actually being there.
My job was to assist in this novel form of journalism as we sought to cover new crises in
that Arab Spring. Another uprising had begun in Syria in March. By the summer of 2011,
what had started as peaceful protest was descending into an increasingly bloody civil war.
The Syrian regime made it almost impossible for foreign correspondents to enter the
country, but the mounting death toll and huge political implications meant that the story
could not be ignored. I was tasked with covering it, without leaving NPR’s offices, just as
Carvin had covered Tunisia and Egypt.
Despite the regime’s restrictions, social media and the Internet were awash in videos,
photos, and reports of events happening inside Syria. Most came from activists or
ordinary citizens who opposed the regime of Bashar al-Assad. It was clear their reports
could not merely be taken at face value. I had to figure out which sources were supplying
reliable information—verifiable images and accurate descriptions of what was going on
thousands of miles and several time zones to the east of where I was working.
As with any beat, the first step was to learn all I could—to read widely about the conflict
and Syrian history, making note of the names and roles of newsmakers, and talking to a
range of people with expertise on Syria and how the conflict was unfolding. I began with
a community I already knew: Syrian bloggers and activists, whom I had come to know in
the mid-2000s when I created an English-language blog, Saudi Jeans, in Saudi Arabia. At
that time, blogging was just taking off in the Middle East and we were a small band of
Arab writers who quickly became acquainted—first online, and later in person at
conferences across the region and abroad.
There were only a handful of Syrian bloggers, and those I knew were critical of the
regime; in fact, as their blogging put them at increased risk of arrest, most were forced to
leave the country.
I reached out to these writers, and our early conversations helped me understand the
Syrian protests and pointed me to other sources of reliable reporting. Social networks,
particularly Twitter, had become popular, and by following and engaging with
people deeply immersed in the story, I could keep up to date with developments and their
significance.
But if you don’t start out with a network of contacts, as I did, how do you find sources
and determine their reliability? There are several reporting techniques and tools that can
help.
Tactic 1: Follow the Experts
Regardless of where the story is, there are a few foreign editors who are always worth
following on Twitter because they excel at keeping up with international news and
analysis across the globe. These include Politico’s Blake Hounshell^{\href{#endnotes}{10}} and BuzzFeed’s
Miriam Elder.^{\href{#endnotes}{11}} Robert Mackey of the New York Times does an excellent job curating
social media content on all kinds of breaking news, including international stories.^{\href{#endnotes}{12}} The
Electronic Frontier Foundation’s Jillian C. York follows Internet freedom issues around
the world,^{\href{#endnotes}{13}} while Zeynep Tufekci of the University of North Carolina offers smart
insights on how technology intersects with social phenomena like protest and political
change, a common theme in many contemporary conflicts.^{\href{#endnotes}{14}}
On a specific story, though, you also need to follow the tweets of local reporters,
activists, and citizens on the ground. Look for names in the initial research you do. One
of the noticeable patterns on Twitter is that whenever a major breaking news event begins
overseas, experts and foreign news geeks who already know that region immediately start
retweeting sources with whom they are familiar. That doesn’t mean everything they cite
is reliable, but their sources are worth a good look and an investment of shoe leather (or
thumb taps) to check out what they report.
In the Syrian uprising, I quickly found and followed experts on the country like Randa
Slim,^{\href{#endnotes}{15}} Andrew Tabler,^{\href{#endnotes}{16}} Rime Allaf,^{\href{#endnotes}{17}} and Joshua Landis.^{\href{#endnotes}{18}} I also looked for journalists
who were covering the story like Rania Abouzeid,^{\href{#endnotes}{19}} Liz Sly,^{\href{#endnotes}{20}} Maisa Akbik,^{\href{#endnotes}{21}} and Javier
Espinosa.^{\href{#endnotes}{22}} I followed activists inside and outside the country. Some were known, like
Rami Jarrah^{\href{#endnotes}{23}} and Mohammed Al Abdullah.^{\href{#endnotes}{24}} Others tweeted anonymously, but over
time I was able to verify the reliability of those like BSyria^{\href{#endnotes}{25}} and THE_47th.^{\href{#endnotes}{26}} Eventually,
I pulled all these sources and others together into a Twitter list that helped me follow the
story over time.
Tactic 2: Know Local Digital Customs
It’s important to know which sites and platforms are popular in the countries you’re
covering and how these sites are used there. Twitter and Facebook are both widely used
in the Middle East, for example, while WeChat and LINE are more popular in Asia.
Within the Middle East, there are differences from country to country. In Egypt, for
example, the government and the military typically use their Facebook pages, and not
their official sites, to make announcements and release statements, while activists and
journalists rely on Twitter.
Compare that with Saudi Arabia, where the culture is more oriented to the spoken word
than to writing and reading. Twitter is by far the most-used platform for breaking and
discussing the country’s news, followed by mobile apps like WhatsApp. Saudi Arabia has
more Twitter users per capita than any other country, and the local twittersphere buzzes
with everything from breaking news, to debates on religion’s role in public life, to the
minutiae of football matches.
Then there’s China, where the government blocks Facebook and Twitter. The domestic
microblogging platform Weibo is a lively social media platform and venue for dissent,
though WeChat has stolen some of its following because it’s less subject to government
censorship—at least so far.^{\href{#endnotes}{27}}
Understanding these local differences among platforms and who uses them is one key to
finding reliable sources and information. Another key is seeking good search tools.
Tactic 3: Use Data Management Tools
On the tools side, creating Twitter lists and filtered searches focused on specific events
and beats in apps like TweetDeck can help you follow important stories as they evolve.^{\href{#endnotes}{28}}
This is essential on busy news days, when events move so quickly that it’s difficult to
identify relevant tweets among the flood of information (which can be exacerbated by
commercial spammers trying to take advantage of news events for marketing purposes).
For example, when anti-government protests began in Ukraine in February of 2014, the
news verification startup Storyful posted a list of Twitter accounts to follow to keep up to
date with the latest from there. Storyful has similar lists for different countries and beats,
from topics like South Africa^{\href{#endnotes}{29}} to wildfires.^{\href{#endnotes}{30}}
A tool that can help assess a Twitter source’s trustworthiness is Twiangulate, a site that
allows you to find the common followers of two Twitter users.^{\href{#endnotes}{31}} For example, you can
compare the followers of a source you already know and trust with the followers of a new
source you just found. If the two share many of the same followers, then there is a good
chance that the new source is one worth following—though you’ll still want to do
additional verification. While following the 2011–2012 protests in Kuwait, for example, I
used Twiangulate to cross-reference some new Twitter users I came across before
deciding whether to follow or retweet them.^{\href{#endnotes}{32}} I used the same method to vet new sources
while following the news of protests by the Shia minority in eastern Saudi Arabia.
Once you have found and connected with new sources, contact them via Twitter direct
message, communicate with them using IM apps, use Skype to video chat, or call them
by mobile phone to talk. I relied on old-fashioned telephone technology when covering
Syria because the Internet infrastructure in the country is weak, especially in the hottest
conflict zones. In these cases, I used digital media to ask the sources for their phone
numbers, then called them and recorded interviews. Later we used some of these audio
conversations in NPR broadcasts and on the website.
NPR has advanced equipment to record audio, but new smartphones can get the job done
when needed. News organizations can configure apps such as Report-IT to allow their
sources to record their end of the interviews and then send the audio via the Internet’s
File Transfer Protocol (FTP). Sources can also use an app like DropVox to record sound,
and then upload it to Dropbox.
Verifying Information
In April of 2013, the annual Boston Marathon was interrupted when two pressure-cooker
bombs exploded, killing three people and injuring over two-hundred and fifty others. In
the hours following the attack, a tsunami of news, rumor, and speculation flooded social
media as authorities worked to identify suspects.
On the Web, users of the popular online community Reddit led their own crowdsourcing
effort to find the suspects. That effort wrongly named two people. One of them, a college
student who had disappeared earlier, was eventually found dead in the Providence River.
He had no connection whatsoever to the bombing, but Reddit’s mistake ensured that his
family suffered through a nasty, hostile media frenzy—a tragic tale and a sober reminder
that verification is at the heart of all good journalism, regardless of the information’s
source.^{\href{#endnotes}{33}}
Digital technology has made that especially true during breaking news events, when false
information and rumors can travel rapidly and widely. Several tools can help journalists
assess what they’re watching and seeing.
Tactic 4: Corroborate Before You Go Public
One verification tool is Storyful’s free Chrome extension, which can quickly and
efficiently search multiple social networks with preset filters to find the best image,
video, and text results around certain keywords and news events.^{\href{#endnotes}{34}} Cross-referencing
information posted on these different sites can be useful to verify this information.
Some resources are available online to help journalists learn more techniques to ensure
that you get the story right before sharing it. One of them is the Verification Handbook by
the European Journalism Centre (EJC).^{\href{#endnotes}{35}} It provides step-by-step guidelines for using
content found on social media during emergencies and crisis situations. Another resource
is the Citizen Evidence Lab by Amnesty International, where you can find tutorials and
case studies about using video and photos for reporting.^{\href{#endnotes}{36}}
Context is also key, especially in the case of verifying the authenticity of photos and
video. Is there evidence they have been altered? Can you identify the location by seeing
buildings or landmarks? Misidentified, altered, or faked images seem to emerge from
every major international story these days; journalists who don’t take care to identify
them can end up spreading deception and destroying their professional credibility. If you
see a picture and don’t know its sourcing, don’t trust—verify.
Here’s how you can try to do that: Every photo taken by a digital camera or a mobile
phone comes with a set of data known as ``Exchangeable image file format,'' or Exif for
short. This data can include information about the type of camera that captured the
photograph. Exif data can also sometimes reveal the last piece of software used to save
the image, as well as the image’s location and time stamp. Tools like Jeffrey’s Exif
Viewer can help you in the verification of photos, as it displays date, time, and location
data for the photos that people share on social media.^{\href{#endnotes}{37}} If the Exif data is not available,
use Google Reverse Image Search^{\href{#endnotes}{38}} or TinEye^{\href{#endnotes}{39}} to check the past life of the photo.
In the case of video, make sure its timestamp matches the weather reported for the date
and time. If there are people speaking in the video, note their accents. If you are unsure,
ask people who follow you on Twitter for help.^{\href{#endnotes}{40}} Here’s an example of Andy Carvin
using Twitter for verification, and one of the responses he received.
Carvin wanted to confirm the location of where a video had been taken. He asked his
followers for help, and several of them volunteered. After he received several answers, he
retweeted what he believed was the most informed one.
One of the best known practitioners of this kind of open-source investigation is Eliot
Higgins, who writes the Brown Moses blog about the Syrian conflict. Higgins’ focus is on
munitions used in the war. He came to the subject with no prior expertise. His reporting
combines obsessive scrutiny of online videos and photos posted by Syrians with appeals
for input from his Facebook and Twitter followers. (``Anyone know what the guy in this
video is saying?'')^{\href{#endnotes}{41}}
Of course, the best sources for verification come from a network of reliable contacts built
up before a story breaks. In the heat of fast-paced news, the pressure is high and mistakes
can be harmful, even dangerous. In these situations, journalists need to pause and ask
how essential or urgent is the piece of information, and how can they take more steps to
verify it before it gets published. Always ask, have we made every effort to reach
authoritative sources? Are there more ways to reach out?
Tactic 5: Pick Up The Telephone
During the Middle East Respiratory Syndrome virus outbreak in Saudi Arabia in 2014, I
found plenty of information on Twitter and Facebook that contradicted the government’s
insistence that the situation was under control. On social media, though, there was a sense
of panic. My challenge was to verify the tweets and posts I was reading.
I started by combing through both platforms, looking for doctors and nurses working in
hospitals that were treating victims. When I found names, I picked up the telephone and
called them. While digital media provided me with leads, traditional technology
connected me to firsthand, knowledgeable sources who could speak in detail about the
situation in their hospitals. By talking to healthcare workers and hearing their accounts, I
had a better understanding of the situation on the ground and presented a more accurate
take on the crisis than what Saudis were learning from either the government or social
media rumor mills.
On Facebook, you can search specifics using the site’s powerful social graph search. To
use it, go to the search box and type a description, such as: ``Nurses who work in Jeddah,
Saudi Arabia'' or ``Doctors who work in Riyadh, Saudi Arabia.'' You can even focus the
search further, as I did by choosing a specific hospital where the outbreak happened:
``Nurses who work at King Fahad Hospital in Jeddah.'' The site will return a list of users
who match this description, and then you can contact them by sending them a Facebook
message asking to talk. As for Twitter, you can search the website for keywords and
phrases. In my case I started with words that indicated panic about the disease outbreak,
and from there started looking for users who worked in hospitals and asked to speak with
them.
This takes time and effort, and it doesn’t always yield results. But in the case of the virus
story, some health workers did respond to my requests and they offered important context
for some of the stories we wrote about the disease for the Wall Street Journal.^{\href{#endnotes}{42}}
Working Transparently
Be honest. Don’t lie. Attribute properly. These may sound like basic rules every
journalist should follow, but they are doubly important in an age when information
assaults us from all sides.
Journalists should not distribute information that they cannot verify. If they want to use
social media in the process of vetting, they should clearly acknowledge that the
information is unverified and they are seeking help or comment from others who may be
able to confirm or debunk it. Pay attention to the language used by others when they
report the news, and be careful how you convey information, because words matter.^{\href{#endnotes}{43}} Be
transparent about how you acquired information and how you went about verifying it.
When I was covering the Syrian uprising for NPR in 2012, I found myself correcting
some of my followers who thought I was tweeting from battlefields in the Middle East,
asking me about specific things or wishing me to stay safe.^{\href{#endnotes}{44}} ``Thanks,'' I would reply to
them, ``but I’m actually tweeting from the comfort of my office in Washington, D.C.''
Being transparent about your location and sourcing is essential for credibility and trust.
Moreover, journalists should make it easy for the audience to find out more about them
and their backgrounds by providing links to their bios and previous work.
Finally, as a reporter, don’t be afraid to show your human side on social media. Tweeting
a link to a music video you like or posting a photo you took during a walk on a sunny
summer day reminds your audience that you have a life beyond work, just as they do. It
can help them relate to you, and to your more serious work.
Tactic 6: Own Your Errors
Even after making every effort to identify sources, verify information, and double-check
copy, mistakes can still happen. And when reporting and publishing are done in real time,
as in the case of live-tweeting, the chances of error are higher.
Journalists should be readily accountable for their mistakes, acknowledging and
correcting them as quickly and as clearly as possible. If the error appeared on Twitter,
Facebook, and your own website, the correction must also appear in all of these places.
Responsibility doesn’t stop with your own errors. If you are a reporter following a story
on Twitter and you see misinformation being published, you need to jump into the
conversation and correct it. Unfortunately, Twitter does not make it easy to correct errors
and rumors, but journalists should do their best to fight the spread of misinformation—
especially on their beats and within the stories they are covering.^{\href{#endnotes}{45}} People who have
retweeted or shared an erroneous post you made may not see your subsequent correction
unless you reach out to them individually. Check for retweets, likes, or re-shares and then
make sure everyone who spread your error gets a copy of the correction.
That’s harder to do if a tweet has been widely retweeted. One method for dealing with
Twitter mistakes is to issue the correction as a reply to the original erroneous tweet, as
Slate did when it posted a photo of Javier Bardem in a tweet about Russian president
In June of 2014, a BBC
journalist tweeted a
screenshot of a tweet
from someone he
identified as ``Saudi
Human Rights Minister.''
I replied to him with a
correction: Saudi Arabia
doesn’t have a human
rights ministry.
The journalist later
posted a correction, but
while his original tweet
was retweeted more than
seventy-five times, his
correction was only
retweeted four.
Securing Your Work
Journalists have a responsibility to keep sources safe from the risks posed by using the
Internet for communications. An application like Skype may be a great way to reach
someone in a far-off place, but it may also give a repressive government or other hostile
party an opportunity to spy on that source.
Tactic 7: Keep Your Sources Safe
The Journalist Security Guide, published by the Committee to Protect Journalists, in
essence advises that we ``think before contacting.''
Many journalists feel that what they are doing is largely transparent, and that they
have nothing to hide. But think about the dangers to sources if the information they
have provided to you was more widely known. What may seem innocuous personal
information to you might be incriminatory to others.^{\href{#endnotes}{47}}
Information security issues are addressed at length in CPJ’s handbook,^{\href{#endnotes}{48}} and in reports
from Internews^{\href{#endnotes}{49}} and other organizations that outline practical steps for using encryption,
avoiding hacks, and steering clear of suspicious links to protect sourcing, your own
equipment and information. However, technology and software change all the time so it
is very important to think about these changes and how to keep sources—and yourself—
safe.
When I started blogging in Saudi Arabia in 2004 I only used my first name, out of fear
for my security. But a few weeks after I began, I realized that my security effort was
deeply flawed: I’d already revealed other information about myself on the blog, such as
where I was going to school, making it fairly easy for someone to figure out all my
credentials. Eventually, I decided to use my full name and real identity, because I thought
it would encourage people to take the blog more seriously.
As my site grew more known, both inside and outside Saudi Arabia, it became clear that
having a high profile actually offered some protection. If authorities arrest an anonymous
blogger, it can go unnoticed. But if you are well known and get arrested, people will
know—possibly leading to protests on your behalf and pressure on the government to
release you. While I tried to keep a balanced tone on the blog, despite my efforts the
government eventually contacted my family to express its displeasure with what I was
writing. I kept on blogging anyway, until I moved to the United States a few months later
to study journalism.
Now, back in Saudi Arabia, I report for the Wall Street Journal. I have the support and
resources of a major international media organization, which enables me to tackle more
serious stories than I could as an individual blogger. But it also means my work is more
carefully scrutinized by authorities than the blog, Saudi Jeans, ever was. I remind myself
regularly that when reporting digitally, my sources may be operating in a more dangerous
legal, technical, and security context than I am. Before making contact, I consider my
responsibility for knowing what risks they face just for communicating with me.
Tactic 8: Remember That Everything Changes
The skills foreign correspondents employ and the tools they use may have changed over
the years, but the essence of the job remains the same. Even as the world becomes
increasingly connected, journalists continue to provide an important service in explaining
global events and giving us a better understanding of what is happening, why, and what
the consequences are.
But just as the telex replaced the telegraph, and cell phones are displacing landlines, new
tools will emerge in international reporting. Whatever form those tools may take,
whatever benefits they may offer, correspondents need to use them with the same
principles that apply today: Verify everything before you publish; be transparent about
your work; keep yourself and your sources secure.
\chapter{The Foreign Desk in Transition: A Hybrid Approach to Reporting From There—
and Here}
By Anup Kaphle
When the Washington Post’s new owner, Jeff Bezos, met the newsroom for the first time
in October of 2013, he spent more than an hour fielding questions from a staff curious to
gauge the Amazon founder’s plans for the one hundred thirty seven-year-old newspaper.
During the session, Bezos mentioned two recent Post stories that he found particularly
intriguing.
The first was a human-interest feature on the death of a bar bouncer, the kind of richly
descriptive narrative that has been a Post hallmark for decades. But Bezos’ other favorite
was something of a surprise: a 2,800-word piece published in the Post’s foreign affairs
blog, headlined ``^{\href{#endnotes}{9}} questions about Syria you were too embarrassed to ask.''^{\href{#endnotes}{50}}
Conceived and reported in Washington by a Post digital journalist, and written for an
online audience, the Syria piece addressed readers in a conversational tone rarely, if ever,
used in traditional foreign reporting. If you ``aren’t exactly sure why Syria is fighting a
civil war, or even where Syria is located,'' wrote blogger Max Fisher, ``this is the article
for you.'' No need to feel embarrassed, he continued. ``What’s happening in Syria is really
important, but it can also be confusing and difficult to follow even for those of us glued
to it.''^{\href{#endnotes}{51}}
Even without the newsroom plug from Bezos, ``^{\href{#endnotes}{9}} questions'' was already grabbing
attention inside and outside the Post. In the two months after it first appeared on
WorldViews, the blog that is one of the paper’s main experiments in international digital
journalism, ``^{\href{#endnotes}{9}} questions'' got over five-million page views. Compare that to the potential
audience for a top international story in the printed newspaper: About 475,^{\href{#endnotes}{000}}
subscribers receive it, and on a good day it might get another 100,^{\href{#endnotes}{000}} page views online.
So, is ``^{\href{#endnotes}{9}} questions'' the future of international news: breezy, digital-first, and written by
someone in an office thousands of miles from the scene? Perhaps the best answer is, it’s a
piece of the hybrid that is foreign news reporting today at the Post and other mainstream
organizations committed to serious international coverage.
Traditional foreign correspondents remain at the heart of that hybrid, filing vivid,
detailed, firsthand reporting from the field. Now, they also fill frequent online updates on
major breaking news. But in-house journalists who don’t leave the office are also a part
of the foreign report. In at least two legacy newsrooms, the Washington Post and the New
York Times, these digital journalists are daily contributors, aggregating, curating, and yes,
doing original reporting—for WorldViews at the Post, and for the New York Times’ Open
Source column by Robert Mackey and Watching Syria’s War.
I am a digital foreign editor at the Post, where we call WorldViews a blog. The Times
labels Open Source a column, while the URL for Watching Syria’s War uses the term
project. The varied labels give some hint at the uncertainty that hangs over traditional
foreign desks in this transitional age. Each of those digital features offers interesting,
innovative reporting. Each is part of mainstream’s push to expand international reporting
beyond the traditional foreign-correspondent model and appeal to more online readers.
But whether these new models will prove as durable as the traditional correspondent
depends on factors that foreign desks didn’t have to worry about in the past: Can they
draw a strong, sustainable audience? And can they play a part in solving the economic
crisis that has caused so many mainstream organizations to axe their foreign bureaus?
Shuttered Bureaus
A report published by the American Journalism Review (AJR) in 2011 found that at least
twenty U.S. newspapers and other media outlets had eliminated all of their foreign
bureaus since AJR first conducted a similar census in 1998.^{\href{#endnotes}{52}} And even in traditional U.S.
newsrooms that continue to maintain foreign bureaus around the world, the number and
size have shrunk dramatically in recent years.
Among newspapers, the Wall Street Journal still has the largest international reporting
division, with correspondents in forty-nine countries, followed by the New York Times
with reporters in twenty countries. Wire services are much larger. The Associated Press
maintains bureaus in seventy-nine countries,^{\href{#endnotes}{53}} while Bloomberg has correspondents in
seventy-three countries.^{\href{#endnotes}{54}}
That’s down from a time, fifteen years or so ago, when the paper kept twenty bureaus
staffed across the globe. But the international staff is no longer limited to correspondents
based in foreign bureaus. In the mid-2000s, as many newsrooms sought new ways to
engage online audiences, the Post hired videojournalist Travis Fox for a new kind of
Web-only foreign reporting—new at least for a traditional newspaper.
Fox traveled around the world producing long, feature-length pieces for the Web.^{\href{#endnotes}{56}} His
stories were fully reported, beautifully shot videos. Such work is costly to sustain,
though, and like many other print organizations experimenting with video, the Post
determined the cost was not yielding the advertising or the online audience it had
expected. After Fox’s departure in December of 2010, wire services became the main
source of video for foreign stories on the Post’s website, with some contributions from
the paper’s own correspondents in the field.
Enter the Blog
By 2012 online innovators at mainstream media were focused on blogs as a key to
attracting new audiences looking for specialized material or faster dispatches on breaking
news. The Post and other big newspaper websites were hosting dozens of blogs on a wide
range of topics. In international affairs, some blogs focused on a single country (China,
India) or a particular conflict (Iraq, Syria). At the Post, writers from the paper and its
sister publication, Newsweek, were paired to discuss world news and foreign affairs in a
blog called PostGlobal that is no longer active.
The international blog that eventually became WorldViews began as an experiment in
2012. ``We wanted to offer readers an opportunity to consume foreign news in a different
way,'' said Douglas Jehl, the paper’s foreign editor, ``one intended to complement the
remarkable work being delivered by our foreign correspondents around the new way.''
Post correspondents were encouraged to contribute to the new blog. But not many leaped
at the idea. ``I felt some skepticism about writing for the foreign blog at first,'' said Kathy
Lally, the Post’s former bureau chief in Moscow. ``Not philosophical questions but
practical ones: How much time it would require was the main question.''
Lally’s reaction has been a common one wherever mainstream media have informed staff
reporters—including foreign correspondents—that their jobs now included writing for the
Web. Unlike the daily paper, delivered just once in twenty-four hours, the Internet never
sleeps. The push to move from a legacy schedule to a 24/^{\href{#endnotes}{7}} one inevitably meets
resistance. ``The habits and traditions built over a century and a half of putting out the
paper are a powerful, conservative force as we transition to digital,'' noted a recent
internal New York Times report on newsroom innovation.^{\href{#endnotes}{57}}
That observation can apply equally to other legacy media, like at the Post where foreign
correspondents’ reactions to the new blog in 2012 initially boiled down to this: ``You’re
asking us to do more work, for no additional pay.'' (Correspondents get no compensation
for blog contributions.) So, in its earliest days, the Post’s international blog depended on
fairly sporadic field reports, supplemented by Web producers working in Washington. It
was not the most auspicious start.
That changed with the hiring of Max Fisher, the blog’s first full-time staff writer, who
arrived in September of 2012 just two months before the experimental blog was to
officially launch as WorldViews. Fisher had never been a foreign correspondent and did
not travel for his pieces, but he wrote daily about breaking news abroad. His sources
included Post foreign correspondents (feeding information to Fisher was far less trouble
than writing an additional story), as well as other news sites, social media, and video
from public sources like YouTube.
Fisher developed a facility for synthesizing analysis from public data and previously
reported stories—all while remaining in Washington. The result could be both serious
and entertaining, like this piece debunking widespread rumors that Kim Jong Un had fed
his uncle to hungry dogs.^{\href{#endnotes}{58}} Although at times sensational, and occasionally controversial,
Fisher probably became best known for explainer posts like ``^{\href{#endnotes}{40}} maps that explain the
world,''^{\href{#endnotes}{59}} which were drawing a tremendous number of readers to the blog, and thus, to
the Washington Post. In 2013, the Post had over fifty blogs, and WorldViews ranked
among the top five in page views. Fisher left the blog in 2014, but two full-time writers
continue to staff it. ``The conversational, explanatory tone that WorldViews employs has
proven to be enormously appealing, by being timely, smart and fun, all at the same time,''
said foreign editor Jehl, who worked as a traditional correspondent for nineteen years,
reporting from nearly forty countries. Jehl tells his correspondents that there is not much
difference between what they’ve done traditionally and what the blog demands: short,
small pieces told with a distinctive voice.
At the heart of the digital transition, though, is this essential factor in building a global
audience: speed. In the past, a foreign correspondent typically faced one daily deadline.
Today, the idea of having an entire day to report a breaking news story sounds luxurious,
as then-Moscow correspondent Kathy Lally explained in an email interview in April of
2014. ``The other day I covered Vladimir Putin’s annual televised, phone-in question-andanswer
session with the Russian nation,'' she wrote. ``It went on for four hours. I filed a
short story after the first hour and missed some things he was saying while I was writing
and filing.''
Lally went on to write the main story that led the Post’s website and the next day’s
paper.^{\href{#endnotes}{60}} Meanwhile, at the blog, WorldViews published even more dispatches, covering
both the quirky^{\href{#endnotes}{61}} and the newsy^{\href{#endnotes}{62}} items of the speech in close to real time. These were
written by WorldViews bloggers in Washington with email feeds from Lally in Moscow.
On a breaking news story, that kind of multiple filing, by both Lally and the bloggers, is
essential to grab readers who want to know, right now, what’s happening. The blog offers
a platform to publish a story, even if it’s still fragmented and developing.
When Israel launched airstrikes near Damascus last May, for example, YouTube videos
like this^{\href{#endnotes}{63}} became a primary and immediate source for news. In the old newspaper model,
the Post and other publications would have worked on a story about this attack for the
next day’s edition. But in the hybrid newspaper-digital model of today, the Post’s Beirut
bureau put together a story that incorporated the YouTube video, which was already
widely circulating on Twitter, added some reporting context and posted it within hours.^{\href{#endnotes}{64}}
Videos posted on the Internet by activists^{\href{#endnotes}{65}} showed a huge fireball erupting on
Mount Qassioun, a landmark hill overlooking the capital on which the Syrian
government has deployed much of the firepower it is using against rebelcontrolled
areas surrounding the city.^{\href{#endnotes}{66}}
The increased emphasis on speed evokes fears among traditional newsroom editors, who
see the need to file and publish fast as a compromise to accuracy. It doesn’t have to be. A
successful news operation can do both: Post a few paragraphs of news based on what the
reporter knows and then gradually add to it throughout the day. It’s what wire services
have done for decades—though today, in traditional print newsrooms, it’s known as
digital-first reporting.
Over time, the Post’s foreign correspondents have become more active contributors to
WorldViews. Today almost every journalist in the paper’s fifteen bureaus contributes
ideas for posts to the bloggers in Washington. Or they may pass on stories or viral videos
that are big news in the countries where they are reporting. Some embrace the
opportunity the blog offers to tell stories in a more informal voice.
Raghavan’s stories describing wars and conflicts from Sana’a to Baghdad have appeared
on the front pages of the Washington Post many times. But in this blog entry following
the harrowing day in Nairobi, he wrote in first person, connecting with readers on a more
personal level, while still describing the horror on the ground—a mission that a
traditional article format doesn’t provide.
Defining the Future
As a digital foreign editor for the Washington Post, one of my roles is to encourage and
guide our foreign correspondents in the pivot toward digital-first journalism. Earlier this
year, when Will Englund was covering the Maidan anti-government protests in Ukraine,
we asked him to capture some video from the scene.
Englund is sixty-one and has a Pulitzer Prize for investigative reporting. He and Lally, his
wife, shared the Moscow bureau job at the time, which included covering the unrest in
Ukraine. Englund had done video before from Nagorno Karabakh, a long-running
conflict dating back to the Soviet era, but he was far from an experienced videographer.^{\href{#endnotes}{68}}
In Kiev, he used a Canon point-and-shoot camera that the Post gives to all of its foreign
correspondents. It helped, he said, that the Maidan protests were ``always extremely
photogenic.'' He filed the audio recording of his narration separately, and the sound and
images were put together by one of the dozens of video editors the Post has hired in
recent years to boost its online video presence. Englund’s video from Kiev conveyed the
Lally and Englund, who now work as Post editors in Washington, both became important
contributors to WorldViews while they worked overseas. Still, both miss the ``good old
days'' of traditional newspaper foreign correspondence. ``It can be satisfying to be quick
with a story, but it’s not terribly rewarding,'' said Englund. ``And being enslaved by the
Web hugely reduces our ability to explore and dig and do the other acts essential to
quality journalism.''
Englund suggested that the Post leave more international breaking news coverage to wire
service reports, giving the paper’s foreign correspondents time to explore deeper stories.
``I believe the Post would be more valuable, and readable, if it moved away from hourly
hard news, rather than trying to stay on top of it,'' he said.
The debate on the shape of the mainstream foreign desk in the digital era is an ongoing
one. Right now, the Post and other newsrooms are working with a hybrid blend of
traditional correspondents and in-house bloggers, urging them to cooperate and
complement each other’s work to create a fresh, constantly updated foreign report. But
even with the increased U.S. and international audience drawn by WorldViews, it’s
impossible to say whether this formula—or something not yet tried—will be the longterm
foreign-desk model.
A recent move at the New York Times signaled a rethinking of the blog approach there.
The paper has ended or merged more than half of its blogs this year, including two with
an international focus: India Ink and The Lede. India Ink was less than three years old and
relied on freelance contributions, as well as postings by the paper’s correspondents in the
country. The Lede often focused on breaking international stories. Those are now being
covered in a new feature called Open Source, written by Robert Mackey, who previously
wrote The Lede.^{\href{#endnotes}{70}} Open Source is akin to a blog, though it is not identified in the
directory as one,^{\href{#endnotes}{71}} and Mackey’s reporting for it draws on social media updates from
Twitter, Facebook, and YouTube videos.
In an interview in June, Times assistant managing editor Ian Fisher said, ``We’re going to
continue to provide bloggy content with a more conversational tone. We’re just not going
to do [it] as much in standard reverse-chronological blogs.''^{\href{#endnotes}{72}}
The Times’ change may reflect a growing trend: that most digital readers find content
through search and social, rather than by seeking out the homepage of dedicated
platforms. Yet, even digitally native publications such as Quartz^{\href{#endnotes}{73}} and Buzzfeed^{\href{#endnotes}{74}} have
launched features focused on India, and the Wall Street Journal continues to maintain
several of its international blogs, including ones that report on China^{\href{#endnotes}{75}} and India.^{\href{#endnotes}{76}}
These blog experiments are reminders that much in the world of digital media remains
just that: experimentation. They are also evidence of just how much international
reporting has evolved—from an era only a few years ago when the emphasis was on rich,
resource-intensive multimedia storytelling, to a time when newsrooms are struggling to
find less costly ways of engaging a wider audience to meet their advertising goals.
The Digital Natives
The statistics on closings of traditional foreign bureaus are grim, but they do not tell the
whole story of foreign news coverage. While mainstream bureaus have closed, digitally
native sites like GlobalPost have filled some of the gap. GlobalPost was created in 2009
to cover international news for a largely U.S. audience, with sixty-five full-time or
freelance correspondents filing regularly from around the world.
Over the past year or so, at least three other digital natives launched expansions into
foreign news coverage: Huffington Post, BuzzFeed, and VICE, each of which has hired
journalists whose role looks a lot like that of a traditional foreign correspondent.
Perhaps the most surprising of these newcomers is BuzzFeed, a site once known
primarily for features such as ``The 30 Most Important Cats of 2010.'' BuzzFeed’s newly
minted commitment to covering hard news from around the world has already made it a
competitor, with in-depth international reporting from Caracas, Kiev, and other 2014 hot
spots. With more than 160-million unique visitors, BuzzFeed is among the top ten mostvisited
news websites in the United States, though it’s not clear how many of those
visitors are reading the international dispatches. In mainstream newsrooms some have
treated the site’s foray into foreign reporting with a certain disdain.
Peter Preston, former longtime editor of The Guardian, is one skeptic. ``There’s nothing
wrong with angles, twists, listicles,'' he wrote in The Guardian. ``If they encourage
readers, that’s great. If they make money, that’s great too. But they are not salvation for
battling reporters in the depths of Africa, doing stories that matter to them and their
communities.''^{\href{#endnotes}{77}}
Preston’s critique makes BuzzFeed’s foreign desk sound almost as frivolous as its cat
videos. It’s not. When it announced it would cover foreign news, the site hired respected
ex-Guardian Moscow correspondent Miriam Elder, who then recruited reporters to cover
Egypt, Syria, Russia, Ukraine, and other current news centers. Following a fresh fifty
million-dollar investment in August, Buzzfeed has said it plans to open offices in Japan,
All of this begins to sound like new media copying legacy media—at the same time that
legacy foreign desks are trying to adapt to the new world of digital. ``I’m not sure there is
much difference—at the end of the day, the reporting we do is a lot more traditional than
I think a lot of people would expect,'' Miriam Elder told me. ``It’s about making and
meeting sources, making phone calls, finding the news, breaking the news.''
There are fundamental differences, of course. BuzzFeed began life as a purveyor of viral
entertainment, and fluff and sensation still dominate its homepage (you’re far more likely
to access its international reporting from your social feeds). But as a digital native,
BuzzFeed definitely has an edge in solving the biggest conundrum in the new world of
journalism: How do you attract an audience?
Social-friendly headlines are one key to audience engagement. The Post and other
mainstream newsrooms are still catching up to BuzzFeed, Upworthy, and other digital
natives that know how to maximize clicks and shares with head-turning headlines. The
Post now shares detailed visual presentations with national and international
correspondents,^{\href{#endnotes}{79}} to show how lead-ins that read like social headlines can grab attention.^{\href{#endnotes}{80}}
(``Headlines are the new nut grafs'' is a line we hear frequently at digital workshop
sessions for our reporters.)
Also key to answering how to attract an audience is the need to rethink the traditional
U.S. newspaper definition of audience. With few exceptions, papers were local
institutions, serving the community where they published. The Internet lets us reach well
beyond traditional print circulation areas, but it doesn’t tell us how best to do that. ``If we
are going to continue to expand our readership, as we must do in a digital world, growing
those national and international audience will be crucial to our success,'' said Jehl, the
Post’s foreign editor.
Again, digital natives like BuzzFeed may have, at least, a psychological advantage in
their instinctively conscious understanding that audiences are now global. That’s led
BuzzFeed to a new approach in assigning beats. Globalization and the Internet mean that
``someone who lives in an urban center in Russia or Uruguay or Vietnam can have more
in common with each other than with other people in their own countries,'' said Elder. So
instead of making all foreign beats based on geography, Elder has created some ``thematic
beats,'' like international women’s rights. Women’s correspondent Jina Moore literally
travels the world to write about how women became players in Rwanda’s politics,^{\href{#endnotes}{81}} about
Brazil’s decision to pay reparations for maternal death,^{\href{#endnotes}{82}} and about abuse in Iraqi
prisons.^{\href{#endnotes}{83}}
The thematic approach is key to engaging a broad global audience, said BuzzFeed deputy
foreign editor Paul Hamilos, in an interview with Journalism.co.uk. ``You’re not going to
grow a news organization if you only think of your English language readers in your
home country,'' he said.^{\href{#endnotes}{84}}
In the ``good old days,'' mainstream outlets like the New York Times and the Washington
Post set the agenda for foreign news in the United States. In the digital era, their authority
as agenda setters is shared with others, and some of the digital natives may end up
showing them important new paradigms for foreign reporting.
The newcomers still have a lot to prove, though. Their commitment to news, unlike that
of mainstream media, is a new phenomenon. Will they still be in the foreign news
business a decade from now? It’s hard to say—just as it’s impossible to predict the shape
of the traditional newsroom’s foreign desk ten years down the road.
\chapter{David Versus Goliath: Digital Resources for Expanded Reporting—and Censoring}
By Burcu Baykurt
At the height of the 2013 anti-government protests in Istanbul’s Gezi Park, most of
Turkey’s controlled, mainstream media committed mass self-censorship. None drew
more derision for timidity than CNN Turk, which became the target of sardonic jokes for
its decision to air a documentary on penguins at the height of the domestic crisis.
Anyone with an Internet connection and the desire to know what was really going on,
though, could easily switch to one domestic news source for independent reporting:
140journos, a counter-media startup that operated in relative obscurity for a year and a
half before the protests began.
Before Gezi, 140journos founders Engin Önder, Cem Aydoğdu, Oğulcan Ekiz, and Safa
Soydan set out to bridge the wide information gap between those who witness news
events in Turkey and those who produce mainstream journalism there. They composed
journalistic news reports and sent out one hundred forty-character tweets about events
that mainstream media largely ignored: protests by leftist, conservative, or LGBTQ
groups; and reports from courtrooms where credentialed journalists were not granted
entrance. ``We are not journalists and we don’t have a large network yet to say we
represent citizens,'' Önder told me when I first met him in 2012 in the news service’s
shabby basement office, housed inside a four-hundred-year-old Istanbul building.
Then the Gezi protests began. In the information vacuum left by mainstream media,
Önder and his colleagues gained a loyal following by showing their ability to sift through
thousands of tweets around the clock, curate and verify social media content, bury
unfounded rumors, and publish what turned out to be reliable accounts from the protests.
With a reputation for veracity and an impressive list of new followers post-Gezi, Önder
acknowledged with a smile, ``We are all journalists now.''
The triumph of 140journos over media repression in Turkey may seem like a captivating
David and Goliath tale—a narrative we’ve seen elsewhere where social media has
sometimes circumvented power to deliver vital independent reporting to audiences in
China, Iran, and other authoritarian regimes. But as in China, Iran, and elsewhere, the
victory for digital journalism in Turkey has turned out to be severely limited.
During the Gezi Park demonstrations, Prime Minister Tayyip Erdoğan declared social
media ``the worst menace to society.''^{\href{#endnotes}{85}} State controls and surveillance, which have
effectively cowed Turkey’s traditional media for years, soon shifted focus from
newspapers and television to social media and other online information sources. In
February of 2014, for example, the Turkish government tightened its grip on the Internet
by requiring service providers to retain user information for up to two years and hand it
over to regulators on demand.^{\href{#endnotes}{86}} A month later, Twitter was banned^{\href{#endnotes}{87}} for a two-week span
shortly after Erdoğan announced in a rally that he would wipe out social media without
regard for what the international community had to say about it. 88
Examples abound of repressive states manipulating new technologies to curb independent
speech in the digital sphere, from Libya’s shutdown of the Internet^{\href{#endnotes}{89}} during widespread
protests in 2011 to Twitter’s charge that the Venezuelan government blocked images on
the service during 2014’s unrest there.^{\href{#endnotes}{90}} Even more democratic-based governments have
contemplated restrictions on digital communications in moments of crisis—as British
Prime Minister David Cameron considered banning suspects from social media during
the 2011 riots in the United Kingdom.^{\href{#endnotes}{91}}
Government shutdowns and blockages cut the flow of information to domestic audiences,
to foreign correspondents covering major international stories, and to global audiences
accessing news about breaking events directly from social media. They are a clear threat
to Internet freedom. Even when governments do not explicitly interfere with online
content, they can stifle free expression with policies that keep an eye on metadata.^{\href{#endnotes}{92}} In
collecting information about where, when, and with whom people communicate,
metadata—which includes every detail except the content—can reveal a lot, thereby
posing a serious potential threat to source confidentiality. The United States, for example,
justifies its controversial surveillance programs as a security tool, not a weapon for
censorship. But the potential for misuse of surveillance can have a chilling effect on
journalism, as documented in a 2014 report by Human Rights Watch and the American
Beyond overt government censorship and surveillance programs, a third concern for
journalists are Internet governance policies that allow governments to pressure service
providers and services, such as Twitter and Facebook, to block reporting posted on their
platforms. Such actions can turn private companies into Internet censors on behalf of the
government.
If the Internet is an effective tool for resistance, it is increasingly clear that it is also an
effective tool for restriction and surveillance. As sociologist Zeynep Tufekci said, ``The
design of today’s digital tools makes the two inseparable.''^{\href{#endnotes}{94}}
Digital Repression and Censorship
When the Internet first became commercially available in the 1990s, its users faced very
few restrictions. In a world where, as media critic A.J. Liebling put it, ``Freedom of the
press is guaranteed only to those who own one,'' suddenly millions could own the means
of transmitting information and ideas to a mass audience. The Internet, a decentralized
platform, offered a level playing field for pesky bloggers, freelance journalists, or
activists to reach audiences around the world without seeking the institutional approvals
of governments or big media companies.
Two decades later, however, almost every government in the world regulates or restricts
online communications in varying degrees.^{\href{#endnotes}{95}} As more governments have begun to use
those regulatory powers to monitor, censor, or shut down digital communications, the
notion of the Internet as a freewheeling forum immune to repression has dimmed.
At a relatively low level of harassment, governments are simply using new digital
platforms to apply some of their traditional tactics—delivering intimidating messages by
email, for example, instead of by phone. Now, though, a threatening message can be sent
at once, en masse. Such was the case when journalists, activists, and others present at
anti-government demonstrations in Kiev in early 2014 received this mobile text: ``Dear
subscriber, you are registered as a participant in a mass disturbance.''^{\href{#endnotes}{96}}
Digital technology can also add new dimensions to smear campaigns, an effort attributed
to Iranian authorities that created fake blogs to discredit Iranian journalists living in exile
in early 2013.^{\href{#endnotes}{97}} The smear might also include a call to digital action, as happened in
Turkey when an official attacked journalist Selin Girit, who works for the BBC, after she
tweeted a comment from a Gezi protester calling for an economic boycott for six
months.^{\href{#endnotes}{98}} Ankara’s pro-government mayor attributed the quote to Girit herself in Twitter
posts, calling her a British agent and urging his more than 700,^{\href{#endnotes}{000}} followers to spam
Girit’s account. Many apparently responded; the reporter said her work during the busiest
days of the Gezi protests was severely compromised by spam attackers.
Then there are more formal strategies for repression, like new laws or decrees put in
place by governments seeking to constrain online communications. Some of these
censorship efforts are obvious and blunt: Turkey, for example, has at different times
blocked access to Twitter and YouTube. Iran, China, and North Korea have blocked
Twitter on and off,^{\href{#endnotes}{99}} and Google reported that its products and services, including
YouTube, have been disrupted around the world more than one-hundred times since
2009.^{\href{#endnotes}{100}}
Another form of online harassment, of concern to both individual journalists and news
organizations, is the distributed denial of services (DDoS) attack. Essentially, this
involves overloading a website with traffic—usually generated by bots—so that it cannot
be accessed for some period of time. According to the national security documents leaked
by Edward Snowden in 2013, the British intelligence agency shut down Internet chat
rooms used by activists associated with Anonymous by mounting a DDoS attack.^{\href{#endnotes}{101}} The
case marked ``the first Western government'' known to have engaged in this particular
form of restriction, according to NBC News.^{\href{#endnotes}{102}}
The Anonymous revelation was unusual; in most cases it is almost impossible to confirm
when or whether a government has employed DDoS as a way to shut down websites.
More typical was a case involving Ustream.tv, a popular website that allows protesters
from around the globe to stream video from their phones. In 2012, the site’s system came
under a large DDoS attack designed to target a Russian who had used Ustream.tv to
broadcast information about Moscow protests against President Vladimir Putin.^{\href{#endnotes}{103}}
Although Brad Hunstable, cofounder and CEO of Ustream, publicly protested the attack,
he could not connect it with a particular government or organization.^{\href{#endnotes}{104}}
Surveillance
Edward Snowden’s 2013 leaks revealed that anyone, including journalists and their
sources, may now be a target of government surveillance. While the U.S. government
says its surveillance is aimed at national security, not journalists, that doesn’t protect
reporters from risk, according to Geoffrey King, Internet advocacy coordinator for the
Committee to Protect Journalists. ``By keeping a record of all communications
transactions swept up in its dragnet, and then linking those transactions to content, the
U.S. government could recreate a reporter’s research, retrace a source’s movements, and
even retroactively listen in on communications that would otherwise have evaporated
forever,'' said King.
It seems that even the fear of exposure is enough to affect potential sources’ willingness
to go on the record, added Trevor Timm of the Freedom of the Press Foundation, a
nonprofit group that promotes transparency and digital security in journalism. ``When
sources know that their information is tracked and stored in a database, it makes them
think twice before ever talking to a journalist,'' said Timm. This impacts investigative
reporting perhaps the most, he said. Even when sources request anonymity to avoid
repercussions for leaking documents or other critical material, they’re unconvinced the
journalist can provide it.
New Risks—Nongovernment Actors
The Committee to Protect Journalists cited digital censorship and surveillance by
governments as two examples of ``the profound erosion of freedom on the Internet'' when
it placed ``cyberspace'' on its 2013 ``CPJ Risk List.''^{\href{#endnotes}{105}} The list normally highlights
countries where press freedom is on the decline; by adding cyberspace to the list, the
committee sought to highlight new threats that transcend country boundaries.
As documented by CPJ, the risks go beyond government interference. Tech-savvy protest
collectives, usually referred to as hacktivists, have been known to launch cyberattacks
targeting the online presence of big news organizations, such as Reuters,^{\href{#endnotes}{106}} the
Washington Post,^{\href{#endnotes}{107}} and the Financial Times.^{\href{#endnotes}{108}} Such collectives also track, and
sometimes hack, individual reporters. Vietnamese activist Ngoc Thu, who runs a popular
dissident blog from California where she resides, became the target of a pro-government
cyber collective last year.^{\href{#endnotes}{109}} After invading her blog and personal email, the hackers
posted her photos on the blog, along with defamatory messages. It took Thu a week to
regain control and move her blog to a new address.
Increasing Corporate Regulation
From tweaking their algorithms^{\href{#endnotes}{110}} to documenting global government requests^{\href{#endnotes}{111}} or
building surveillance tools,^{\href{#endnotes}{112}} many private companies—directly or indirectly—restrict,
control, and direct how news is accessed and shared. Yet, as the Electronic Frontier
Foundation’s (EFF) director for International Freedom of Expression Jillian C. York
pointed out, few people are thinking deeply about the implications that corporate actions
have for free speech.^{\href{#endnotes}{113}}
These actions may come in response to government demands, just as they did last year
when Britain announced it would require Internet service providers (ISPs) to block sites
the government deems ``terrorist'' or otherwise dangerous.^{\href{#endnotes}{114}} In many countries, including
the United States, ISPs are sometimes legally required to collect user data in bulk.^{\href{#endnotes}{115}} The
ISPs—mostly telephone and cable companies—say they have no choice but to comply
when state agencies demand user information because their business licenses are subject
to approval by those same agencies.
There are some efforts afoot to increase transparency about government demands.
Vodafone, the world’s second biggest mobile phone company, recently published its first
Law Enforcement Disclosure Report to document the extent of government demands in
the twenty-nine countries where it operates.^{\href{#endnotes}{116}} While most governments need legal
notices or warrants to tap into customers’ mobile communications, Vodafone said there
are six countries where ``the law dictates that specific agencies and authorities must have
direct access to an operator’s network.''^{\href{#endnotes}{117}} The report did not name the countries, but said
that, ``If we do not comply with a lawful demand for assistance, governments can remove
our license to operate.''^{\href{#endnotes}{118}}
Content providers can face similar pressures. Some of them, like Facebook and Twitter,
try to present images of themselves as free speech advocates. But in an area with little
transparency, some of these companies may be taking actions—in response to
government demands—that actually restrict free speech and independent reporting. One
of the most notorious examples is the 2004 case of Chinese journalist Shi Tao, who was
sentenced to ten years in prison on charges of revealing state secrets after Yahoo
complied with a Chinese government request to turn over Shi Tao’s account information
and email content. 119 His email message, which he sent to a U.S.-based website, related a
government mandate to Chinese media organizations to ``downplay'' the upcoming
anniversary of the Tiananmen Square uprising. Shi Tao was released in 2013 after
serving eight years of his sentence.
Rebecca MacKinnon, who closely covered the Shi Tao case and now directs the Ranking
Digital Rights^{\href{#endnotes}{120}} project at the New America Foundation, said companies should take
steps to minimize their complicity in censorship. Still, she acknowledged in a Guernica
feature she wrote, ``If a company were to make a blanket commitment to never comply
with any government censorship or surveillance requests, it would be able to do business
precisely nowhere.''^{\href{#endnotes}{121}} That being said, EFF’s Jillian C. York argued that, in their haste to
comply, companies sometimes go beyond what is legally required in a government
request. ``Facebook hands over data to the Pakistani government even though it has no
offices there and there is no mutual legal assistance treaty between the U.S and Pakistan,''
she said.^{\href{#endnotes}{122}} Another example York cited was Twitter’s decision this year to remove
``blasphemous'' tweets in Pakistan.^{\href{#endnotes}{123}} While the tweets were later restored, she noted that
Twitter did not have to comply in the first place.
using coalitions^{\href{#endnotes}{125}} aimed at reporting instances of online censorship,^{\href{#endnotes}{126}} assessing the
openness of the Internet,^{\href{#endnotes}{127}} and ranking tech companies on their free speech and privacy
practices.^{\href{#endnotes}{128}} Yet many negotiations between governments and companies take place
behind closed doors. In Turkey, for example, soon after Twitter was temporarily banned
in early 2014, Colin Crowell, the company’s head of policy, met with government
officials in Ankara.^{\href{#endnotes}{129}} The details of those meetings were never made public. A week
after the visit, Twitter blocked two accounts that had written about corruption allegations
involving Prime Minister Erdoğan, his family, and associates.^{\href{#endnotes}{130}}
There are no enforceable global guidelines for protecting independent online reporting or
at least forcing more transparency about government demands. Informally, though, one
effort to mitigate these problems involves content providers sending government requests
they receive to the Chilling Effects Project, which creates an archive of legal notices.^{\href{#endnotes}{131}}
Through the website, users can track what kind of content governments seek to censor. It
does not, however, reveal actions the private companies take on their own.
In early 2014, The Atlantic reported that dozens of Syrian opposition websites were shut
down for posting what Facebook deemed to be graphic imagery or calls to violence that
violated its service terms.^{\href{#endnotes}{132}} As the company does not make public its reasons for such
decisions, activists remain unclear about where Facebook draws the line between news
and events that incite violence. In another case, the Facebook page of an alternative news
outlet in Turkey, Ötekilerin Postası (The Others’ Post), was shut down several times last
year. Facebook said the action was triggered by a large number of user complaints. The
company did not explain any further beyond saying the page had violated Facebook’s
community standards. ``What Facebook is doing is censorship: blatantly taking sides and
engaging in digital vandalism,'' said Ötekilerin Postası, whose editors were banned from
posting to the page for thirty days.^{\href{#endnotes}{133}}
In owning the platforms that many people use for news-sharing, debate, and activism,
companies like Google and Facebook are at the helm of deciding who can say what—and
who gets heard. As the big sources of traffic to digital content, these companies filter
online news not based on the public value of a message, but on the potential increase in
user engagement and advertising revenue.^{\href{#endnotes}{134}} They constantly change their algorithms,
which vary country by country; the content that users see on their timelines or as search
results depends on thousands of variables. How the companies manage those algorithms
is not transparent, despite the significant impact they have on how we disperse and read
online news.^{\href{#endnotes}{135}}
In other words, for those who produce and share news reports online, there is a lot to be
done in ``an information environment where the content is ours but the selection is
theirs,'' noted communications scholar Tarleton Gillespie.^{\href{#endnotes}{136}}
Fighting Back—Or, Protecting Yourself?
Edward Snowden’s revelations about widespread surveillance underscored that digital
security is now a press freedom issue. As The Guardian’s Alan Rusbridger put it, ``It’s
essential to be paranoid'' about cyber security in the post-Snowden era.^{\href{#endnotes}{137}} But being
worried isn’t enough. A more robust digital security toolbox and a series of behavioral
changes are imperative on both individual and institutional levels.
Most technology experts agree that designing effective precautions demands that
journalists first define the potential threats they face. Only then can they seek tools to
keep their reporting secure. Unless there are technologists available in the newsroom,
however, journalists are often on their own to recognize the possible digital security risks.
Leaving it to individuals could mean that little gets done to increase security. ``There’s
still hesitancy among many journalists to go through security training and implement it,''
said Jennifer Henrichsen, an independent researcher who conducted an electronic survey
on the safety of online journalists for the United Nations Organization for Education,
Science, and Culture (UNESCO). Even for technologically literate journalists, it is
usually extremely difficult and time consuming to take even the basic security
precautions, she said. And, ``Many journalists may not perceive a direct harm to
themselves or their sources.''
For those who do pursue security training, there are many online tools, such as encryption
software, that are easily accessible and often available at no charge. Susan E. McGregor,
assistant director of the Tow Center for Digital Journalism at Columbia University,
described encryption as ``the process of scrambling and encoding messages'' so that only
someone with the correct key can unlock the original.^{\href{#endnotes}{138}} As leaving no traces online
becomes a wider concern for anyone who reports journalistically in the age of mass
surveillance, secure chat software^{\href{#endnotes}{139}} and Tor for anonymous browsing^{\href{#endnotes}{140}} should become
more commonplace in interviewing and research.
The bad news is, there is no one-size-fits-all approach to security so keeping a
journalist’s work secure can’t be done with the flip of a switch. Within big news
organizations there is a constant search for ways to build more security into internal
communications networks. The Freedom of the Press Foundation—a San Franciscobased
group whose board includes Edward Snowden, Daniel Ellsberg, and journalists
Glenn Greenwald and Laura Poitras—promotes tools for news outlets to evade
surveillance by governments. The SecureDrop141—an open-source whistleblowing
platform to submit anonymous leaks—is one of them, now used by publications such as
The Guardian, the Washington Post, and ProPublica.
It is not only a crackable password or connecting to insecure networks that can put
journalists and their sources at risk, though. Some intelligence agencies, including the
NSA and Britain’s GCHQ, reportedly undermine encryption standards of networks
through bribery, cajoling, and legal processes.^{\href{#endnotes}{142}} Pointing to the overall danger that acts
like these inject into the whole network, CPJ’s Geoffrey King said, ``We should be fixing
things, patching holes rather than just letting those vulnerabilities secretly get worse.''
Moving Forward
When the Turkish government blocked access to Twitter in March of 2014, it took only a
couple of minutes for most users to circumvent the ban. In fact, in the first twelve hours
after the blockage began, there was a thirty-three percent increase in the number of
Twitter messages from Turkey and a seventeen percent increase in the number of unique
users.^{\href{#endnotes}{143}} During the two-week ban, 140journos continued receiving and distributing
citizen reports at a furious pace.
In response, the government ramped up its censorship by blocking Google DNS and
adding YouTube to its list of verboten social media sites.^{\href{#endnotes}{144}} Despite court orders, the ban
on YouTube had still not been lifted as of early April.^{\href{#endnotes}{145}} It wasn’t until late May that the
Constitutional Court of Turkey, the country’s uppermost legal body, ruled in favor of free
speech and ended the block.^{\href{#endnotes}{146}}
The online battle in Turkey reflects a global struggle, in which journalists—traditional or
digital—are constantly engineering new ways to work around repressive tactics, while
governments continue honing their efforts to neutralize, censor, and suppress. These
push-and-pull battles affect not only journalists, but also their sources and audiences
around the globe who seek independent, uncensored reporting in an increasingly digitized
world.
\chapter{A Professional Kinship? Blurring the Lines Between International Journalism and Advocacy}
By Jessie Graham
While Madeleine Bair was a journalism student at the University of California, Berkeley
a few years ago, she landed a coveted summer internship in the Colombia bureau of an
international news agency. Bair imagined that the bureau staff would be in the field
constantly, reporting on dramatic events like the kidnappings and murders of labor
activists and indigenous leaders in rural Colombia. But because of resource constraints
and security concerns, agency staff weren’t able to travel to many regions where big
news was happening.
So instead of learning the ropes of serious field correspondence, she was ``stuck in an
office, reporting on Shakira,'' Bair said. While there, she read More Terrible Than Death,
a 2003 account of the violent fallout of the U.S. war on drugs in Colombia. The book is
based on the kind of serious, analytical reporting that Bair wanted to do. But its author,
Robin Kirk, was a former Human Rights Watch researcher—not a journalist.
Bair returned to Berkeley with an understanding that the career she hoped for was not
necessarily going to take place at a traditional news organization. When she graduated
with her journalism degree in 2011, she took her skills to an advocacy group in Jamaica,
which hired her to produce a series of videos about extrajudicial killings. Later she
landed a fellowship at Human Rights Watch, where she worked on a video about
government involvement in disappearances in Mexico’s drug war. (I was her supervisor.)
Today, Bair works for WITNESS, an international nongovernmental organization that
trains and equips people to use video in their fight for human rights. As curator of the
group’s human rights channel on YouTube, she focuses on collecting, verifying, and
analyzing citizen-generated content from around the world.^{\href{#endnotes}{147}}
Though WITNESS is an advocacy group, not a news outlet, when Bair filed her 2013 tax
return, she wrote ``journalist'' under occupation. ``Wherever I’ve worked, I’ve always
maintained the same integrity or set of standards that I used when reporting for
newspapers or public radio and the skills that I honed in journalism school,'' she said.
``Advocacy journalism has an objective, but that doesn’t mean it’s lesser than some idea
of ‘pure’ journalism.''
Whatever you choose to call Bair—journalist or advocate—her career trajectory shows
how the idealized boundaries between international reporting and advocacy have blurred
over the past decade in the United States. Advocacy groups that once depended on
mainstream media to report the findings of their work can now produce and distribute
reports directly to global audiences. And some media outlets that once drew a clear line
between the work of journalists and the work of advocacy groups are now increasingly
using videos, photos, and other materials from local activists or researchers at global
groups like Human Rights Watch, where I work.
Digital developments are at the heart of this change: specifically, the Internet’s disruption
of traditional economic models in the news business (read: news outlets have slashed
funding for foreign reporting) and the explosion of easy-to-use digital devices now
widely available to advocates, activists, and ordinary citizens all over the globe.
In this environment, new kinds of storytellers are inventing themselves. Many use the
same rigorous, fact-based reporting techniques of American journalism, but they’re also
using their reporting to actively draw conclusions and lobby for change in ways that
traditional journalism in the United States has long shunned.
The change may be most obvious in photography, where a photographer like Marcus
Bleasdale has specialized in documenting one of the main concerns of rights groups: the
human cost of conflict. Bleasdale’s images have appeared in National Geographic, the
New York Times, the Washington Post, and Newsweek, to name a few. But his primary
client and partner over the past few years has been Human Rights Watch.
Wherever he works, Bleasdale said, his methods are classic photojournalism. But he
doesn’t call himself a journalist. ``There’s absolutely no difference in how I operate if I
work for Human Rights Watch, National Geographic, or the New York Times,'' Bleasdale
told me. The disparity lies in how those organizations use the images he makes, and he
strongly prefers working with groups that use his photographs and videos to advocate for
change. ``I started out wanting to be a photographer that focused on human rights,'' he
said. ``And now I’m a human rights activist that uses photography.'' Once he’s made an
image, Bleasdale said, his next question is, ``Who can you put that photo in front of to
make a change?''
Nonprofit News
Nonprofit news used to mean The Associated Press and NPR. Both are clearly news
organizations, but they don’t rely on advertisers for funding. Today, we’re seeing a new
crop of digitally native journalism sites funded by philanthropy and operating as
nonprofits. Some are simply using a new funding source to produce traditional
journalism. Others are using foundation funding to produce or bankroll reporting with a
strong point of view, often focused on a specific premise—like the Marshall Project’s
assertion that the U.S. justice system is broken and needs fixing or FirstLook’s
commitment to exposing violations of privacy and civil liberties carried out in the name
of national security. It’s a digital expression of an old-fashioned, muckraking tradition
and ``a huge change for the better for people who want to go deeper,'' said Dan Gillmor,
director of the Knight Center for Digital Media Entrepreneurship at Arizona State
University.
Paralleling the creation of some point-of-view news enterprises is the growth of
communications efforts by advocacy organizations, such as Human Rights Watch, where
I have worked since 2008. As traditional newsrooms have closed or reduced their foreign
news operations, many of their journalists have moved to fulltime jobs with advocacy
groups tackling international issues. These are the same groups that once depended on
mainstream media to get their research out to a wide audience. They still pursue coverage
in local newspapers and radio, on cable news shows, or placement of a timely op-ed in
the Washington Post. But their in-house communications departments are now likely to
have staff capable of producing their own multimedia packages, which may reach
audiences first via the organization’s own website or through YouTube.
In the week before the start of the 2014 Sochi Olympics, for example, Human Rights
Watch released a video about Russian vigilante groups attacking gay men.^{\href{#endnotes}{148}} News
outlets picked up the story, and Reddit and Buzzworthy helped the video reach three and
a half-million views in a little over two weeks.^{\href{#endnotes}{149}}
Some of the communications staff at Human Rights Watch who work on stories like
these are veterans of the news and documentary world. So are some of the researchers in
the field, like Letta Tayler, senior researcher for terrorism/counterterrorism at Human
Rights Watch and a former foreign correspondent for Newsday.
As a journalist, Tayler covered the U.S. invasion of Iraq, prisoner abuse at Abu Ghraib,
and detainees at Guantanamo Bay, among other post-9/^{\href{#endnotes}{11}} stories. Her focus at Human
Rights Watch is on human rights abuses perpetrated in the name of counterterrorism. She
researches and writes detailed, heavily documented reports like ``Unpunished Massacre:
Yemen’s Failed Response to the ‘Friday of Dignity’ Killings,'' which calls for, among
other actions, international sanctions on Yemeni officials deemed responsible for the
killings of anti-government protestors during the 2011 Arab Spring uprisings.^{\href{#endnotes}{150}} The
same research is often the basis for shorter pieces, without the calls to action, which
Tayler writes for Salon, Foreign Policy, GlobalPost, and others.
``We’re becoming a little more like journalists, and journalists are becoming a bit more
like us,'' said Emma Daly, a former foreign correspondent for the New York Times who is
now communications director at Human Rights Watch. ``We try to use journalistic
techniques. Journalists use nonprofits for funding, and even for specialized reporting.''
The Advocacy Newsroom
Before I came to Human Rights Watch I reported for public radio and other news
organizations for more than a decade, funding some of my international reporting with
grants from nonprofit fellowship programs. I was devoted to journalism and didn’t
imagine I’d ever defect to an advocacy group—until foreign reporting budgets began to
shrink and new jobs like my current one (senior multimedia producer) began to appear at
established nonprofits.
I don’t call the work we do at Human Rights Watch journalism, though our workplace
can sometimes feel like a newsroom. The focus is much narrower than at a mainstream
news organization; it’s exclusively on human rights issues. And while we often pursue
the same stories that mainstream journalists are covering (see the recent New York
Times^{\href{#endnotes}{151}} and Human Rights Watch^{\href{#endnotes}{152}} coverage of mass killings in Tikrit, Iraq), our
criteria for deploying resources gives added weight to working where we can bring about
the most change. That leads to a different emphasis—our reports stress the abuses,
abusers, and international standards more than most journalists would in their stories.
But if you see a Human Rights Watch researcher doing interviews in the field, it won’t
look much different from a foreign correspondent on the job—assuming that
correspondent is reporting on human rights issues and has the time and inclination to
speak with scores of witnesses and survivors. Researchers check these accounts against
other sources—activists, local journalists, country experts, and government officials.
Human Rights Watch increasingly seeks comment from those accused of abuses, just as
journalists do.
When our research is complete, our report is written and our lawyers vet it for
consistency with international law, as well as libel and fairness concerns; our work does
not end with publication. We always issue a clear call for change, usually in the form of
detailed recommendations to governments and international institutions. The Human
Rights Watch website describes the goal like this: ``We work to increase the price of
human rights abuse. The more tyrants we bring to justice, the more potential abusers will
reconsider committing human rights violations.''
With emergency situations happening in places now covered by few foreign
correspondents, we and other advocacy groups often serve as a kind of niche wire service
for breaking news. As fighting erupted in the Central African Republic in 2013, for
example, Human Rights Watch’s Peter Bouckaert^{\href{#endnotes}{153}} and Bleasdale, the photographer,
were among the few foreigners reporting on a sustained basis from the ground. Their
work for Human Rights Watch ran on CNN and ABC, and on the Foreign Policy and
National Geographic websites. Similarly, the breaking news blog at the New York Times
website used tweets from Amnesty International’s Joanne Mariner, including a photo of a
Muslim boy in the Central African Republic who had suffered a machete attack at the
hands of the Christian militia.^{\href{#endnotes}{154}}
``Five years ago NGOs were just starting to experiment with the idea that they were
media providers,'' said Ethan Zuckerman, director of the Center for Civic Media at the
news and information, said Zuckerman, ``they understand they have to be.''
Amnesty International, the International Committee of the Red Cross, and the
International Rescue Committee have created portals that make their content available for
journalists to download upon registering with the site. Human Rights Watch has a similar
system that’s led to increased pickup from mainstream broadcasters—even by major
outlets that first balked at the idea of using an advocacy group’s raw footage and
produced pieces. Among those who now run Human Rights Watch video on air and
online are CNN,^{\href{#endnotes}{155}} the BBC,^{\href{#endnotes}{156}} and Al Jazeera America. 157 Human Rights Watch has
partnerships with BuzzFeed and the European Broadcasting Union, which often
distributes the group’s raw footage to news outlets worldwide. Our videos are not
advertisements for the organization; they include documentary evidence, such as
interviews that Human Rights Watch researchers have done with survivors of human
rights abuses, satellite images, maps, and primary documents. Some media organizations,
on principle, decline video or other material produced by advocacy groups. Even so,
within mainstream newsrooms newer departments—like blogs that curate and aggregate
from many sources—may have greater leeway in using material like ours.
Robert Mackey, a digital columnist for the New York Times Foreign desk,^{\href{#endnotes}{158}} said he
usually deletes email news releases from advocacy groups. But when a story is breaking
far away, he often posts video or tweets from people working on the ground for Human
Rights Watch, Amnesty International, Doctors Without Borders, or local groups like the
activist Egyptian filmmakers at Mosireen.^{\href{#endnotes}{159}} ``It does seem clear that you guys are doing
the work of journalists in a lot of cases,'' Mackey told me. ``It doesn’t matter whether the
person doing the work has a point of view, but whether the work is rigorous and factual.''
Paying the Bills
In recent years, Human Rights Watch has started to make raw footage and produced
pieces available with many of our reports. We train researchers to gather multimedia in
the field and pair them with photojournalists who can help them find visual evidence and
tell visual stories.
Journalists, particularly freelance photographers, who want to report on under-covered
international stories say that because of budget cuts at traditional media outlets they are
now often able to travel only thanks to paid assignments from advocacy or aid groups, or
from grant-giving organizations like the Pulitzer Center on Crisis Reporting. In 2010, for
example, the Pulitzer Center and Human Rights Watch teamed up to send Bleasdale to
follow the trail of the Lord’s Resistance Army (LRA) in Congo, the Central African
Republic, Sudan, and Uganda.^{\href{#endnotes}{160}} The videos they produced had a strong advocacy slant.
In one, victims of the LRA appealed to President Obama to act on commitments to help
neutralize the rebel group.^{\href{#endnotes}{161}} The video won two Webby Awards, an excerpt was
broadcast on the PBS Newshour, and Bleasdale’s still photos from the same assignment
were published in The Telegraph,^{\href{#endnotes}{162}} Le Monde, Newsweek, and Stern.
The Pulitzer Center and Human Rights Watch also hosted an event with Bleasdale to help
draw attention to the story. Jon Sawyer, the Pulitzer Center’s director and a former
foreign correspondent, said he is comfortable with such collaborations. ``I think that every
journalism project worth its salt is advocating something, is trying to persuade you of
something,'' Sawyer said. ``Otherwise, why go out and spend the time and the effort to
master a subject and go out and interview people?''
These partnerships have become economically vital for photographers from agencies like
VII and Magnum, many of whom have been taking assignments from aid and advocacy
groups for more than a decade. Last November, for example, the photojournalist Ed
Kashi produced a video about Syrian refugee youth that was published on Time’s
website.^{\href{#endnotes}{163}} The magazine’s fee was tiny, though, Kashi said. His travel expenses to the
Syrian refugee camps were paid by the International Medical Corps, which was working
in the camps.^{\href{#endnotes}{164}} ``There’s less commissioning, not just less money but also less desire on
the part of magazines for this sort of in-depth, visual storytelling,'' said Kashi, who has
done similar reporting stints with funding from Open Society Foundations and Oxfam.
He is currently working on a project with Human Rights Watch.^{\href{#endnotes}{165}} ``The NGOs and
foundations and nonprofits of the world are really stepping up to pay for, as well as
facilitate, the fieldwork.''
The relationship is symbiotic. Advocacy groups need photos and videos to help them tell
their stories. Journalists want to tell important stories about global issues. The
photographers and videographers who enter into such deals say they usually can strike an
agreement that allows them to retain copyrights for their work. So the group funds the
trip and gets the material it needs, and the journalist is then free to sell his or her material
to news outlets.
Kashi, who came of age when magazines had ample money to commission regular
foreign reporting, said he never imagined he’d be using advocacy and aid groups as a
primary source of funding. He’s still nostalgic for the days when magazines had money
to send him off to cover the world, with few limitations.
But he also said he is comfortable taking assignments from groups like La Isla
Foundation, an organization whose sole focus is on ending an epidemic of a rare kidney
disease among sugar cane workers in Latin America.^{\href{#endnotes}{166}} He’s especially proud of a series
he did on the lingering effects of Agent Orange in Vietnam. That project, funded by the
Ford Foundation, won photo prizes, but more important to Kashi was this result: The U.S.
Congress committed four million dollars to help clean up Agent Orange in Vietnam. ``To
me that is so much more interesting and powerful than, ‘Hey, look, I got the cover of
National Geographic with this picture, but nothing happened,’ '' Kashi said.
Strings Attached?
Other journalists told me they want to keep a clear distance from advocacy groups and
funders. Rebecca Hamilton, a human rights lawyer who is a former Reuters
correspondent and the author of Fighting for Darfur, covered South Sudan for the
Washington Post from 2010–2011 thanks to a Pulitzer Center grant. The center is run by
journalists, but Hamilton worried that the grant’s funder, Humanity United167—which is
run by Ebay’s Pierre Omidyar and his wife—would try to influence her reporting.
That didn’t happen, said Hamilton, who credits Sawyer, the Pulitzer Center director, with
maintaining a clear firewall that prevents funders from dictating the direction the
reporting takes. ``That would be a mess,'' said Sawyer, ``if you had an advocacy
organization that was trying to channel its own point of view through the journalist.''
Hamilton said she’s never worked directly for an advocacy group and she wants to keep
it that way, though she sees overlaps between her work and that of human rights
researchers. ``I think you can make that transition once,'' she said. ``Going back and forth
is difficult.''
But editors at many news outlets—especially those with limited budgets for foreign
reporting—are increasingly open to working with journalists who have received
specialized grants for reporting, or who travel on assignment for advocacy groups or UN
agencies. ``There just isn’t enough money to pay for the journalism that we want to do,''
said Jennifer Goren, assignment editor at the BBC/PRI radio show The World.
Goren does not see a freelancer’s experience with advocacy groups as an automatic
liability. Still, she remains on the lookout for potential conflicts of interest. If a reporter
heads to Africa with money from the Carter Center, for example, The World will not
accept a story about a Carter Center program. But a piece on an unrelated topic would
likely pass muster.
Perhaps the best example I’ve found of the growing overlap between advocacy work and
journalism is that of the veteran foreign correspondent Janine di Giovanni, Middle East
Editor at Newsweek, who recently ended a nine-month contract as a Syria Consultant for
the United Nations High Commissioner for Refugees (UNHCR).^{\href{#endnotes}{168}}
Di Giovanni responded to questions, via email, just before heading off to Iraq to cover
the crisis in Mosul for Newsweek in June. ``I have found nothing unethical, no
dilemmas—other than the time factor, of course... so more an issue of balancing my time
than a moral one,'' she wrote. ``Both organizations—Newsweek and UNHCR—are places
I am very proud to work for.'' Di Giovanni said she never took simultaneous travel
assignments from Newsweek and UNHCR. When she quoted a UNHCR report in a
Newsweek story, she identified herself as an advisor to UNHCR.
Charles Sennott, cofounder of the international news website GlobalPost (where I worked
before coming to Human Rights Watch), said he sees a natural convergence between
advocacy and journalism. ``Any good journalist advocates for human rights,'' said
Sennott, a former foreign correspondent for the Boston Globe. His latest project is
GroundTruth, a nonprofit, investigative arm of GlobalPost that is funded mainly by
foundations like Ford and Henry Luce. GroundTruth doesn’t shy away from taking a
clear view on issues like global income inequality^{\href{#endnotes}{169}} or drones.^{\href{#endnotes}{170}} Sennott calls this work
``social justice journalism.'' Furthermore, he has no problem using work from journalists
on assignment for advocacy or aid groups, as long as these relationships are spelled out
clearly to his readers.
Conventional Definitions Be Gone
Editors weren’t very open to these kinds of arrangements when I graduated from
while I am concerned about how financially untenable and unsafe it is for freelance
journalists to work abroad these days, I’m glad there are now options beyond the
traditional foreign correspondent route, through assignments and jobs with groups like
Human Rights Watch.
Now when recent graduates ask me for advice on reporting internationally, I tell them not
to worry if an outlet or a gig meets the conventional definition of journalism. I suggest
they find work that will allow them to learn about and report on subjects that matter to
them. And that’s what is promising about international reporting today. Journalists are
deciding what matters to them and they’re going after it. Some may be using journalistic
techniques but calling themselves advocates. Either way, the space to do international
investigative reporting, traditionally found within mainstream news outlets, is now just as
likely to be available at a nonprofit startup, through a collective of freelancers, or at an
advocacy group.

\chapter{Endnotes}

1 Columbia University International Newsroom, ``The New Global Journalism,'' 2012,
http://newglobaljournalism.com/.\\
2 R. Cohen, ``A Journalist’s ‘Actual Responsibility,’'' New York Times, 5 Jul. 2009,
http://www.nytimes.com/2009/07/06/opinion/06iht-edcohen.html?_r=1&.\\
3 A. Moaveni, Honeymoon in Tehran (New York: Random House, 2009).\\
4 K. Golnoush Niknejad, ``The Virtual Iran Beat,'' NiemanReports, Nieman Foundation, accessed 15 Sep.\\
2014, http://niemanreports.org/articles/the-virtual-iran-beat/.\\
5 ``Guardian Hosts Tehran Bureau,'' The Guardian, 13 Jan. 2013, http://www.theguardian.com/world/iranblog/
2013/jan/13/guardian-hosts-tehran-bureau-iran.\\
6 J. Sutter, ``Celebrity Death Rumors Spread Online,'' CNN.com, 1 Jul. 2009,
http://www.cnn.com/2009/TECH/07/01/celebrity.death.pranks/,
7 D. Carr, ``The Pressure to Be the TV News Leader Tarnishes a Big Brand,'' New York Times, 21 Apr.\\
2013, http://www.nytimes.com/2013/04/22/business/media/in-boston-cnn-stumbles-in-rush-to-breaknews.\\
html?pagewanted=all&_r=1&.\\
8 I. Katz, ``SXSW 2011: Andy Carvin—the Man Who Tweeted the Revolution,'' The Guardian, 14 Mar.\\
2011, http://www.theguardian.com/technology/2011/mar/14/andy-carvin-tunisia-libya-egypt-sxsw-2011.\\
9 C. Silverman, ``Is This the World’s Best Twitter Account?,'' Columbia Journalism Review, 8 Apr. 2011,
http://www.cjr.org/behind_the_news/is_this_the_worlds_best_twitter_account.php.\\
10 B. Hounshell, @blakehounshell, Twitter, https://twitter.com/blakehounshell.\\
11 M. Elder, @MiriamElder, Twitter, https://twitter.com/MiriamElder.\\
12 R. Mackey, @RobertMackey, Twitter, https://twitter.com/RobertMackey.\\
13 J. York, @jilliancyork, Twitter, https://twitter.com/jilliancyork.\\
14 Z. Tufekci, @zeynp, Twitter, https://twitter.com/zeynep.\\
15 R. Slim, @rmslim, Twitter, https://twitter.com/rmslim.\\
16 A. Tabler, @Andrewtabler, Twitter, https://twitter.com/Andrewtabler.\\
17 R. Allaf, @rallaf, Twitter, https://twitter.com/rallaf.\\
18 J. Landis, @joshua_landis, Twitter, https://twitter.com/joshua_landis.\\
19 R. Abouzeid, @Raniaab, Twitter, https://twitter.com/Raniaab.\\
20 L. Sly, @LizSly, Twitter, https://twitter.com/lizsly.\\
21 M. Akbik, @M_akbik, Twitter, https://twitter.com/M_akbik.\\
22 J. Espinosa, @javierespinosa2, Twitter, https://twitter.com/javierespinosa2.\\
23 R. Jarrah, @RamiJarrah, Twitter, https://twitter.com/Ramijarrah.\\
24 M. Al Abdallah, @Mohammad_Syrica, Twitter, https://twitter.com/Mohammad_Syria.\\
25 BSyria, @BSyria, Twitter, https://twitter.com/BSyria.\\
26 The 47th, @THE_47th, Twitter, https://twitter.com/THE_47th.\\
27 I. Johnson, ``An Online Shift in China Muffles an Open Forum,'' New York Times, 4 Jul. 2014,
http://www.nytimes.com/2014/07/05/world/asia/an-online-shift-in-china-muffles-an-open-forum.html.\\
28 A. Moon, ``Using Tweetdeck as a Newsgathering Tool,'' ABC Australia, 23 Apr. 2013,
http://about.abc.net.au/2013/04/using-tweetdeck-as-a-newsgathering-tool/.\\
29 Storyful, ``South Africa Elections,'' Twitter, https://twitter.com/Storyful/lists/south-africa-elections.\\
30 Storyful, ``Wildfires,'' Twitter, https://twitter.com/Storyful/lists/wildfires.\\
31 Twiangulate, http://twiangulate.com/search/.\\
32 ``How to Use Triangulate for Twitter Research,'' Folk Media, 2 Jul. 2010, http://folkmedia.org/how-touse-
twiangulate-for-twitter-research/.\\
33 J. Kang, ``Should Reddit Be Blamed for the Spreading of a Smear?,'' New York Times Magazine, 25 Jul.\\
2013, http://www.nytimes.com/2013/07/28/magazine/should-reddit-be-blamed-for-the-spreading-of-asmear.\\
html?_r=1&.\\
34 Storyful, Multisearch Chrome Extension, https://chrome.google.com/webstore/detail/storyfulmultisearch/
hkglibabhninbjmaccpajiakojeacnaf?hl=en.\\
35 C. Silverman, ed., Verification Handbook (Netherlands: European Journalism Centre, May 2014),
accessed at http://verificationhandbook.com/book/.\\
36 Citizen Evidence Lab, Amnesty International, http://citizenevidence.org/.\\
37 J. Friedl, ``Jeffrey’s Exif Viewer,'' http://regex.info/exif.cgi.\\
38 Google, Inside Search, http://www.google.com/insidesearch/features/images/searchbyimage.html.\\
39 TinEye, Reverse Image Search, https://www.tineye.com/.\\
40 A. Carvin, @acarvin, Twitter, 8 Jun. 2012, https://twitter.com/OmeimaY/status/211087533528977408.\\
41 P. Keefe, ``Rocket Man: How an Unemployed Blogger Confirmed That Syria Had Used Chemical
Weapons,'' The New Yorker, 25 Nov. 2013, http://www.newyorker.com/magazine/2013/11/25/rocket-man-
2.\\
42 E. Knickmeyer and A. Omran, ``Deadly Virus’s Spread Raises Alarms in Mideast,'' Wall Street Journal,
13 Apr. 2014,
http://online.wsj.com/news/articles/SB10001424052702303887804579499831393801054?mg=reno64-wsj.\\
43 A. Carvin, ``When Reporting Breaking News, Words Matter—And Sometimes Languages, Too,'' Andy
Carvin’s Waste of Bandwidth, 11 Oct. 2013, http://www.andycarvin.com/?p=1967.\\
44 A.Omran, @ahmed, Twitter, 5 Feb. 2012, https://twitter.com/ahmed/status/166370339041976320.\\
45 Z. Tufekci, ``Habemus Erratum: How Twitter Could Help Fight the Spread of Errors,'' Technosociology
blog, 14 Mar. 2013, http://technosociology.org/?p=1213.\\
46 C. Silverman, ``Slate’s Good Strategy for Correcting Errors on Twitter, Elsewhere,'' Poynter, 4 Mar.\\
2014, http://www.poynter.org/latest-news/regret-the-error/235681/slates-good-strategy-for-correctingerrors-
on-twitter-elsewhere/.\\
47 F. Smyth, Journalist Security Guide, Committee to Protect Journalists, accessed 8 Sep. 2014 at
https://www.cpj.org/reports/2012/04/journalist-security-guide.php
48 F. Smyth, ``Information Security,'' Journalist Security Guide, Committee to Protect Journalists, Ch. 3,
accessed 8 Sep. 2014 at https://www.cpj.org/reports/2012/04/information-security.php.\\
49 M. Aryal and S. Tennyson, ``SpeakSafe: Media Workers’ Toolkit for Safer Online and Mobile Practices,''
Internews, 2012, https://speaksafe.internews.org/uploads/files/Internews_SpeaksafeToolkit.pdf.\\
50 M. Fisher, ``^{\href{#endnotes}{9}} Questions About Syria You Were Too Embarrassed to Ask,'' WorldViews blog,
Washington Post, 29 Aug. 2013, http://www.washingtonpost.com/blogs/worldviews/wp/2013/08/29/9-
questions-about-syria-you-were-too-embarrassed-to-ask/.\\
51 Ibid.\\
52 P. Kumar, ``Foreign Correspondents: Who Covers What,'' American Journalism Review, Dec./Jan. 2011
iss., accessed at http://ajrarchive.org/article.asp?id=4997.\\
53 Global Bureaus/Correspondents, The Associated Press, accessed 18 Sep. 2014 at
http://www.ap.org/Images/bureau-list_tcm28-12238.pdf.\\
54 Bloomberg Facts, Bloomberg, accessed 18 Sep. 2014 at http://www.bloomberg.com/now/bloombergfacts/.\\
55 Foreign Bureaus, Washington Post, accessed 18 Sep. 2014 at
http://www.washingtonpost.com/world/foreign-bureaus/.\\
56 Videojournalist Travis Fox, Washington Post, accessed 18 Sep. 2014 at
http://www.washingtonpost.com/wp-dyn/content/linkset/2005/04/30/LI2005043000376.html.\\
57 A. Wills, ``The Full New York Times Innovation Report,'' New York Times Company, 24 Mar. 2014,
accessed at http://www.scribd.com/doc/224608514/The-Full-New-York-Times-Innovation-Report.\\
58 M. Fisher, ``No, Kim Jong Un Probably Didn’t Feed His Uncle to 120 Hungry Dogs,'' WorldViews blog,
Washington Post, 3 Jan. 2014, http://www.washingtonpost.com/blogs/worldviews/wp/2014/01/03/no-kimjong-
un-probably-didnt-feed-his-uncle-to-120-hungry-dogs/.\\
59 M. Fisher, ``^{\href{#endnotes}{40}} Maps That Explain the World,'' WorldViews blog, Washington Post, 12 Aug. 2013,
http://www.washingtonpost.com/blogs/worldviews/wp/2013/08/12/40-maps-that-explain-the-world/.\\
60 K. Lally, ``Putin’s Remarks Raise Fears of Future Moves Against Ukraine,'' Washington Post, 17 Apr.\\
2014, http://www.washingtonpost.com/world/putin-changes-course-admits-russian-troops-were-in-crimeabefore-
vote/2014/04/17/b3300a54-c617-11e3-bf7a-be01a9b69cf1_story.html.\\
61 T. Rupar, ``Putin Indicates He’s Not Planning on Taking Over Alaska,'' WorldViews blog, Washington
Post, 17 Apr. 2014, http://www.washingtonpost.com/blogs/worldviews/wp/2014/04/17/putin-says-hes-notplanning-
on-taking-over-alaska/.\\
62 A. Taylor, ``Putin Told Edward Snowden That Russia Doesn’t Use Mass Surveillance on its Citizens.\\
Here’s a Reality Check,'' WorldViews blog, Washington Post, 17 Apr. 2014,
http://www.washingtonpost.com/blogs/worldviews/wp/2014/04/17/putin-told-edward-snowden-that-russiadoesnt-
use-mass-surveillance-on-its-citizens-heres-a-reality-check/.\\
63 ``'' Free Qudsaya’s Channel, YouTube, 4 May
2013, https://www.youtube.com/watch?v=f_j8ID-m1pU&feature=youtu.be.\\
64 S. Haidamous and L. Sly, ``Syrian Report: Israel Bombs Outskirts of Damascus for Second Time in
Recent Days,'' Washington Post, 4 May 2013, http://www.washingtonpost.com/world/middle_east/israellaunches-
second-airstrike-in-syria-targeting-weapons-shipment/2013/05/04/cdccddc0-b4c4-11e2-9fb1-
62de9581c946_story.html.\\
65 ``2013/5/5'' D.C. Pulsation, YouTube, 4 May 2013,
http://www.youtube.com/watch?v=e84pVGsP6YU&amp;feature=youtu.be.\\
66 Haidamous and Sly, Washington Post.\\
67 S. Raghavan, ``At Nairobi Mall, Covering a War Zone Too Close to Home,'' WorldViews blog,
Washington Post, 22 Sep. 2013, http://www.washingtonpost.com/blogs/worldviews/wp/2013/09/22/atnairobi-
mall-covering-a-war-zone-too-close-to-home/.\\
68 W. Englund, ``The Future of Nagorno-Karabakh,'' Pulitzer Center, 3 Aug. 2011,
http://pulitzercenter.org/video/nagorno-karabakh-azerbaijan-armenians-war-territory.\\
69 W. Englund, ``Behind the Barricades in Ukraine,'' Washington Post, 20 Feb. 2014,
http://www.washingtonpost.com/posttv/world/behind-the-barricades-in-ukraine/2014/02/20/28d35a88-
9a60-11e3-9900-dd917233cf9c_video.html.\\
70 R. Mackey, ``The Case for ISIS, Made in a British Accent,'' Open Source, New York Times, 20 Jun. 2014,
http://www.nytimes.com/2014/06/21/world/middleeast/the-case-for-isis-made-in-a-britishaccent.\\
html?_r=^{\href{#endnotes}{1}}
71 New York Times, Blog Directory, accessed 22 Sep. 2014 at
http://www.nytimes.com/interactive/blogs/directory.html.\\
72 A. Beaujon, `` ‘Almost Half’ of the NYT’s Blogs Will Close or Merge,'' Poynter, 25 Jun. 2014,
http://www.poynter.org/latest-news/mediawire/256936/almost-half-of-the-nyts-blogs-will-close-or-merge/.\\
73 Quartz India, Quartz, accessed 22 Sep. 2014 at http://qz.com/india/.\\
74 BuzzFeed, Latest on India, accessed 22 Sep. 2014 at http://www.buzzfeed.com/tag/india.\\
75 China Realtime blog, Wall Street Journal, accessed 22 Sep. 2014 at http://blogs.wsj.com/chinarealtime/.\\
76 India Realtime blog, Wall Street Journal, accessed 22 Sep. 2014 at http://blogs.wsj.com/indiarealtime/.\\
77 P. Preston, ``One Big Reason Why Buzzfeed Might Never Win a Pulitzer Prize,'' The Guardian, 19 Apr.\\
2014, http://www.theguardian.com/media/2014/apr/20/buzzfeed-pulitzer-prize-never-one-big-reason.\\
78 M. Isaac, ``^{\href{#endnotes}{50}} Million New Reasons BuzzFeed Wants to Take Its Content Far Beyond Lists,'' New York
Times, 10 Aug. 2014, http://www.nytimes.com/2014/08/11/technology/a-move-to-go-beyond-lists-forcontent-
at-buzzfeed.html.\\
79 T. Herrera and R Kellett, ``Wapo.st/heds,'' Washington Post presentation, accessed 22 Sep. 2014 at
https://docs.google.com/presentation/d/1QV8LvyeFYnM1pEORzclonmwMjNHEK9HwV72rtzIB1RQ/htm
lpresent.\\
80 A. Odysseus Patrick, ``For a Century, the Mansion Sat Above Sydney Harbor. Then China’s Nouveau
Riche Arrived,'' Washington Post, 17 Jun. 2014, http://www.washingtonpost.com/world/for-a-century-themansion-
sat-above-sydney-harbor-then-chinas-nouveau-riche-arrived/2014/06/17/c6c228aa-f1b3-11e3-
914c-1fbd0614e2d4_story.html.\\
81 J. Moore, ``How Rwandan Women Got Their Power,'' BuzzFeed, 11 Apr. 2014,
http://www.buzzfeed.com/jinamoore/how-rwandan-women-got-their-power#3gyjtxb.\\
82 J. Moore, ``Brazil Just Became the First Country Ever to Pay Reparations for a Maternal Death,''
BuzzFeed, 26 Mar. 2014, http://www.buzzfeed.com/jinamoore/in-global-first-brazil-pays-reparations-formaternal-
death#3gyjtxb.\\
83 J. Moore, ``Abuse, Including Rape, is Standard in Iraq’s Prisons, Report Says,'' BuzzFeed, 6 Feb. 2014,
http://www.buzzfeed.com/jinamoore/rape-is-standard-in-iraqs-prisons-report-says#3gyjtxb.\\
84 A. Reid, ``How BuzzFeed World is Making the Outlet a ‘One-Stop Shop,'' journalism.co.uk, 15 Apr.\\
2014, http://www.journalism.co.uk/news/how-buzzfeed-world-is-making-the-outlet-a-one-stop-shop-
/s2/a556470/.\\
85 W. Oremus, ``Turkish Prime Minister Blames Twitter for Unrest, Calls It ‘the Worst Menace to Society,’
'' Future Tense blog, Slate, 3 Jun. 2013,
http://www.slate.com/blogs/future_tense/2013/06/03/turkey_protests_prime_minister_erdogan_blames_twi
tter_calls_social_media.html.\\
86 C. Letsch, ``Turkey Pushes Through New Raft of ‘Draconian’ Internet Restrictions,'' The Guardian, 6
Feb. 2013, http://www.theguardian.com/world/2014/feb/06/turkey-internet-law-censorship-democracythreat-
opposition.\\
87 E. Dockterman, ``Turkey Bans Twitter,'' Time, 20 Mar. 2014, http://time.com/32864/turkey-bans-twitter/.\\
88 ``Turkey’s Erdogan Threatens to Ban Twitter,'' Aljazeera, 20 Mar. 2014,
http://www.aljazeera.com/news/europe/2014/03/turkey-erdogan-threatens-ban-twitter-
2014320165956732467.html.\\
89 ``Libya Internet Shut Down Amid Protests, Later Restored,'' Huffington Post, 18 Feb. 2011,
http://www.huffingtonpost.com/2011/02/18/libya-internet-shut-down-_n_825473.html.\\
90 S. Frier, A. Kurmanaev, and P. Laya, ``Venezuelans Blocked on Twitter as Opposition Protests Mount,''
Bloomberg, 14 Feb. 2014, http://www.bloomberg.com/news/2014-02-14/twitter-says-venezuela-blocks-itsimages-
amid-protest-crackdown.html.\\
91 J. Halliday, ``David Cameron Considers Banning Suspected Rioters From Social Media,'' The Guardian,
11 Aug. 2011, http://www.theguardian.com/media/2011/aug/11/david-cameron-rioters-social-media.\\
92 J. Ball, ``NSA Stores Metadata of Millions of Web Users for Up to a Year, Secret Files Show,'' The
Guardian, 30 Sep. 2013, http://www.theguardian.com/world/2013/sep/30/nsa-americans-metadata-yeardocuments.\\
93 ``US: Surveillance Harming Journalism, Law, Democracy,'' Human Rights Watch, 28 Jul. 2014,
http://www.hrw.org/news/2014/07/28/us-surveillance-harming-journalism-law-democracy.\\
94 Z. Tufekci, ``Is the Internet Good or Bad? Yes.'' Matter, 12 Feb. 2014, https://medium.com/matterarchive/
is-the-internet-good-or-bad-yes-76d9913c6011.\\
95 OpenNet Initiative, Country Profiles, https://opennet.net/research/profiles.\\
96 T. Lopez, ``How Did Ukraine’s Government Text Threats to Kiev’s EuroMaidan Protesters?,'' Future
Tense blog, Slate, 24 Jan. 2014,
http://www.slate.com/blogs/future_tense/2014/01/24/ukraine_texting_euromaidan_protesters_kiev_demons
trators_receive_threats.html.\\
97 S.K. Dehghan, ``Iran Creates Fake Blogs in Smear Campaign Against Journalists in Exile,'' The
Guardian, 24 Jan. 2013, http://www.theguardian.com/world/2013/jan/24/iran-fake-blog-smear-campaignjournalist-
bbc.\\
98 ``BBC Reporters ‘Intimidated’ by Turkey,'' BBC, 24 Jun. 2013, http://www.bbc.com/news/world-europe-
23036924.\\
99 D. Liebelson, ``MAP: Here Are the Countries That Block Facebook, Twitter, and YouTube,'' Mother
Jones, 28 Mar. 2014, http://www.motherjones.com/politics/2014/03/turkey-facebook-youtube-twitterblocked.\\
100 Google, Transparency Report, accessed 8 Sep. 2014,
http://www.google.com/transparencyreport/traffic/disruptions/.\\
101 M. Schone, et al., ``Exclusive: Snowden Docs Show UK Spies Attached Anonymous, Hackers,'' NBC
News, 4 Feb. 2014, http://www.nbcnews.com/feature/edward-snowden-interview/exclusive-snowden-docsshow-
uk-spies-attacked-anonymous-hackers-n21361.\\
102 Ibid.\\
103 I. Lunden, ``Ustream Says Russian Citizen Journaist Is The Focus Of Today’s DDoS Attack,''
TechCrunch, 9 May 2012, http://techcrunch.com/2012/05/09/ustream-says-russian-citizen-journalist-is-thefocus-
of-todays-ddos-attack/.\\
104 B. Jacobs and A. Nemtsova, ``Did Russian Authorities Shut Down Ustream?,'' The Daily Beast, 9 May
2012, http://www.thedailybeast.com/articles/2012/05/09/did-russian-authorities-shut-down-ustream.html.\\
105 M. Taal, ``CPJ Risk List: Where Press Freedom Suffered,'' Committee to Protect Journalists, Feb. 2014,
http://cpj.org/2014/02/attacks-on-the-press-cpj-risk-list-1.php.\\
106 P. Mosendz, ``Syrian Electronic Army Hacks Reuters (Again),'' The Wire, Atlantic Media, 23 Jun. 2014,
http://www.thewire.com/technology/2014/06/syrian-electronic-army-hacks-reuters-again/373215/.\\
107 C. Timberg, ``Hackers Break Into Washington Post Servers,'' Washington Post, 18 Dec. 2013,
http://www.washingtonpost.com/business/technology/hackers-break-into-washington-postservers/
2013/12/18/dff8c362-682c-11e3-8b5b-a77187b716a3_story.html.\\
108 H. Mance, ``FT Hacked by Syrian Electronic Army,'' Financial Times, 17 May 2013,
http://www.ft.com/intl/cms/s/0/7a091972-bef3-11e2-a9d4-00144feab7de.html#axzz35gSQvyUw.\\
109 C. Brummitt, ``Vietnam’s ‘Cyber Troops’ Take Fight to U.S., France,'' Associated Press, 20 Jan. 2014,
http://bigstory.ap.org/article/vietnams-cyber-troops-take-fight-us-france.\\
110 K. Wagner, ``Facebook Tweaks News Feed for More ‘High Quality’ Content,'' Mashable, 2 Dec. 2013,
http://mashable.com/2013/12/02/facebook-news-feed-high-quality-content/.\\
111 Google, Transparency Report, accessed 8 Sep. 2014, http://www.google.com/transparencyreport/.\\
112 ``Ethiopia Uses Foreign Kit to Spy on Opponents—HRW,'' BBC News, 25 Mar. 2014,
http://www.bbc.com/news/world-africa-26730437.\\
113 J. Cyork, ``Social Media Has Been Privatised. Why Do We Treat It as a Public Space?,'' Voices blog,
New Statesman, 11 Jun. 2014, http://www.newstatesman.com/internet/2014/06/social-media-has-beenprivatised-
why-do-we-treat-it-public-space.\\
114 J. Garside, ``Ministers Will Order ISPs to Block Terrorist and Extremist Websites,'' The Guardian, 27
Nov. 2013, http://www.theguardian.com/uk-news/2013/nov/27/ministers-order-isps-block-terroristwebsites.\\
115 G. Greenwald, ``NSA Collecting Phone Records of Millions of Verizon Customers Daily,'' The
Guardian, 5 Jun. 2013, http://www.theguardian.com/world/2013/jun/06/nsa-phone-records-verizon-courtorder.\\
116 Vodafone, Law Enforcement Disclosure Report, accessed 8 Sep. 2014,
http://www.vodafone.com/content/sustainabilityreport/2014/index/operating_responsibly/privacy_and_secu
rity/law_enforcement.html.\\
117 Ibid.\\
118 Ibid.\\
119 ``Imprisoned for Peaceful Expression,'' Amnesty International, accessed 8 Sep. 2014,
http://www.amnestyusa.org/our-work/cases/china-shi-tao.\\
120 Ranking Digital Rights, http://rankingdigitalrights.org/.\\
121 R. MacKinnon, ``Playing Favorites,'' Guernica, 3 Feb. 2014,
http://www.guernicamag.com/features/playing-favorites/.\\
122 E. Galperin, ``Pakistan’s 8-Hour Twitter Block Sparks Fears of Future Internet Censorship,'' Deeplinks
blog, Electronic Frontier Foundation, 30 May 2012, https://www.eff.org/deeplinks/2012/05/pakistans-8-
hour-twitter-sparks-fears-future-internet-censorship.\\
123 E. Galperin and J. York, ``Twitter Reverses Decision to Censor Content in Pakistan,'' Deeplinks blog,
Electronic Frontier Foundation, 23 Jun. 2014, https://www.eff.org/deeplinks/2014/06/twitter-reversesdecision-
censor-content-pakistan.\\
124 Telecommunications Industry Dialogue, http://www.telecomindustrydialogue.org/.\\
125 Global Network Initiative, https://globalnetworkinitiative.org/.\\
126 Knight Foundation, Online Censorship.org grant description, accessed 8 Sep. 2014,
http://www.knightfoundation.org/grants/201499906/.\\
127 Knight Foundation, Global Censorship Measurement grant description, accessed 8 Sep. 2014,
http://www.knightfoundation.org/grants/201499904/.\\
128 Knight Foundation, Ranking Digital Rights grant description, accessed 8 Sep. 2014,
http://www.knightfoundation.org/grants/201499907/.\\
129 O. Coskun, ``Turkey Accuses Twitter of ‘Tax Evasion,’ Calls for Local Office,'' Reuters, 14 Apr. 2014,
http://www.reuters.com/article/2014/04/14/us-turkey-twitter-idUSBREA3D0TY20140414.\\
130 ``High-profile Figures Among 52 Detained in Major Corruption Operation in Turkey,'' Hürriyet Daily
News, 18 Dec. 2013, http://www.hurriyetdailynews.com/high-profile-figures-among-84-detained-in-majorcorruption-
operation-in-turkey.aspx?pageID=238&nID=59754&NewsCatID=338.\\
131 Chilling Effects, https://www.chillingeffects.org/.\\
132 M. Pizzi, ``The Syrian Opposition is Disappearing From Facebook,'' The Atlantic, 4 Feb. 2014,
http://www.theatlantic.com/international/archive/2014/02/the-syrian-opposition-is-disappearing-fromfacebook/
283562/.\\
133 E. Güler, ``Facebook Facing Accusations of Censoring Citizen Journalism,'' Hürriyet Daily News, 5 Aug.\\
2013, http://www.hurriyetdailynews.com/facebook-facing-accusations-of-censoring-citizenjournalism.\\
aspx?pageID=238&nID=51996&NewsCatID=374.\\
134 E. Protalinski, ``Facebook’s Referal Traffic Share Grew Over 37\% in Q^{\href{#endnotes}{1}} 2014, Pinterest Was Up 48\%,
Twitter Increased Only 1\%,'' The Next Web, 21 Apr. 2014,
http://thenextweb.com/facebook/2014/04/21/facebooks-referral-traffic-share-grew-37-q1-2014-pinterest-
48-twitter-increased-1/.\\
135 N. Diakopoulos, ``Algorithmic Accountability Reporting: On the Investigation of Black Boxes,'' Tow
Center for Digital Journalism, Dec. 2013, http://towcenter.org/algorithmic-accountability-2/.\\
136 T. Gillespie, ``Facebook’s Algorithm—Why Our Assumptions Are Wrong, and Our concerns Are
Right,'' Culture Digitally, 4 Jul. 2014, http://culturedigitally.org/2014/07/facebooks-algorithm-why-ourassumptions-
are-wrong-and-our-concerns-areright/?
utm_content=bufferf601d&utm_medium=social&utm_source=twitter.com&utm_campaign=buffer.\\
137 A. Chozick, ``The Guardian’s Alan Rusbridger: ‘It’s Essential to Be Paranoid,’ '' New York Times
Magazine, 7 Mar. 2014, accessed at http://www.nytimes.com/2014/03/09/magazine/the-guardians-alanrusbridger-
its-essential-to-be-paranoid.html?_r=1.\\
138 S. McGregor, ``Digital Security and Source Protection for Journalists,'' Tow Center for Digital
Journalism, Jul. 2014, http://towcenter.org/digital-security-and-source-protection-for-journalists/.\\
139 ``ChatSecure: Private Messaging,'' The Guardian Project, https://guardianproject.info/apps/chatsecure/.\\
140 Tor Project, https://www.torproject.org/.\\
141 SecureDrop, Freedom of the Press Foundation, https://pressfreedomfoundation.org/securedrop.\\
142 J. Larson, N. Perlroth, and S. Shane, ``Revealed: The NSA’s Secret Campaign to Crack, Undermine
Internet Security,'' ProPublica, 5 Sep. 2013, http://www.propublica.org/article/the-nsas-secret-campaign-tocrack-
undermine-internet-encryption.\\
143 Somera, @someratr, Twitter, 23 Mar. 2014, https://twitter.com/someratr/status/447672656977207296.\\
144 ``Turkey Widens Internet Censorship,'' Hürriyet Daily News, 22 Mar. 2014,
http://www.hurriyetdailynews.com/turkey-widens-internetcensorship.\\
aspx?pageID=238&nID=63954&NewsCatID=338.\\
145 L. Bicchierai, ``YouTube Remains Blocked in Turkey, Despite Court Order,'' Mashable, 10 Apr. 2014,
http://mashable.com/2014/04/10/youtube-turkey-btk-ignores-court/.\\
146 P. Jenkins, ``Turkey Lifts Two-Month Block on YouTube,'' Time, 4 Jun. 2014,
http://time.com/2820984/youtube-turkey-ban-lifted/.\\
147 WITNESS Channel, YouTube, http://www.youtube.com/user/Witness.\\
148 ``Russia: Gay Men Beaten on Camera,'' Human Rights Watch, YouTube, 3 Feb. 2014,
https://www.youtube.com/watch?v=zMTbFSJ_Tr4&feature=youtu.be.\\
149 C. Hauser, ``Rights Group Releases Video of LGBT Attacks in Russia,'' The Lede blog, New York Times,
4 Feb. 2014, http://thelede.blogs.nytimes.com/2014/02/04/rights-group-releases-video-of-lgbt-attacks-inrussia/?
smid=pl-share
150 L. Tayler, ``Unpunished Massacre: Yemen’s Failed Response to the ‘Friday of Dignity’ Killings,''
Human Rights Watch, 12 Feb. 2013, http://www.hrw.org/reports/2013/02/12/unpunished-massacre.\\
151 M. Shum, et al., ``Video Feature: Surviving an ISIS Massacre,'' New York Times, 3 Sep. 2014,
http://www.nytimes.com/2014/09/04/world/middleeast/surviving-isis-massacre-iraq-video.html?_r=2.\\
152 ``Iraq: Islamic State Executions in Tikrit,'' Human Rights Watch, 2 Sep. 2014,
http://www.hrw.org/news/2014/09/02/iraq-islamic-state-executions-tikrit.\\
153 P. Bouckaert, @bouckap, Twitter.\\
154 C. Hauser, ``Report Details Atrocities in Central African Republic,'' The Lede blog, New York Times, 12
Feb. 2014, http://thelede.blogs.nytimes.com/2014/02/12/report-details-atrocities-in-central-africanrepublic/.\\
155 ``Human Rights Watch: Mass Grave Found in Ukraine,'' CNN Channel, YouTube, 24 Jul. 2014,
https://www.youtube.com/watch?v=OmA4yrVtG8s.\\
156 B. Kendall, ``Nato: Russia Had Illegally Crossed Ukraine’s Border,'' BBC, 29 Aug. 2014,
http://www.bbc.com/news/world-europe-28978065.\\
157 ``Human Rights Watch Accuses Syria Rebels of War Crimes,'' Al Jazeera America, 11 Oct. 2013,
http://america.aljazeera.com/articles/2013/10/11/human-rightswatchsyrianrebelgroupsguiltyofwarcrimes.\\
html.\\
158 R. Mackey, ``Jailed for Protests, Activists in Egypt and Bahrain Turn to Hunger Strikes,'' New York
Times, 4 Sep. 2014, http://www.nytimes.com/2014/09/05/world/middleeast/jailed-for-protests-activists-inegypt-
and-bahrain-turn-to-hunger-strikes.html.\\
159 ``The Mosireen Collective,'' YouTube, https://www.youtube.com/user/Mosireen.\\
160 J. Bavier and M. Bleasdale, ``The Lord’s Resistance Army: The Hunt for Africa’s Most Wanted,''
Pulitzer Center on Crisis Reporting, 7 Sep. 2010, http://pulitzercenter.org/projects/africa/lords-resistancearmy-
hunt-africas-most-wanted.\\
161 ``Joseph Kony—LRA,'' Humans Rights Watch Channel, YouTube, 10 Nov. 2010,
https://www.youtube.com/watch?v=PNL2oyvrJZ0.\\
162 J. Bavier, ``Victims of Uganda’s Lord’s Resistance Army,'' The Telegraph, 18 Feb. 2011,
http://www.telegraph.co.uk/news/worldnews/africaandindianocean/uganda/8331004/Victims-of-Ugandas-
Lords-Resistance-Army.html.\\
163 A. Katz, ``Syria’s Lost Generation: Directors Give Voice to Wars Young Refugees,'' LightBox, Time, 23
Jan. 2014, http://lightbox.time.com/2014/01/23/ed-kashi-julie-winokur-meet-syrias-young-refugees/.\\
164 ``Photojournalist Ed Kashi Documents Life in Syrian Refugee Camps,'' International Medical Corps, 9
Jan. 2014,
https://internationalmedicalcorps.org/2014_january_09_slideshow_edkashi_syrianrefugees#.VBuqCy5dXI
o.\\
165 E. Kashi, ``A Video on Pain Relief in Mexico by Human Rights Watch,'' Kickstarter, 31 May 2014,
https://www.kickstarter.com/projects/1386330672/a-video-on-pain-relief-in-mexico.\\
166 La Isla Foundation, https://laislafoundation.org/.\\
167 Humanity United, http://www.humanityunited.org/.\\
168 http://www.janinedigiovanni.com/.\\
169 C. Sennott, et al., ``The Great Divide: Global Income Inequality and Its Cost,'' GlobalPost, 2014,
http://www.globalpost.com/special-reports/global-income-inequality-great-divide-globalpost.\\
170 C. Sennott, ``The Drone Age: Why We Should Fear Global Proliferation of UAVs,'' GlobalPost, 2014,
http://www.globalpost.com/special-reports/the-drone-age-why-we-should-fear-global-proliferation-uavs.\\